\documentclass[10pt]{article} % use larger type; default would be 10pt
\usepackage[utf8]{inputenc} % set input encoding (not needed with XeLaTeX)

\usepackage{geometry} % to change the page dimensions
\geometry{letterpaper}
\usepackage{graphicx} % support the \includegraphics command and options
\setlength\voffset{-0.2in}

\title{\includegraphics[width=1.55in]{ruddock_crest.png} \\ Procedural Amendment}
\author{Ruddock House}
%\date{} % Activate to display a given date or no date (if empty),

\begin{document}
\maketitle
An Election Committee shall be appointed at the nomination meeting. The Election Committee shall be composed of nine Full House Members: 3 sophomores, 3 juniors and 3 seniors, who shall be randomly selected from a pool of volunteers who are not running for an elected office. In the event that too few members of a specific class volunteer, the remaining members shall be drawn from the other classes.


Paper ballots shall be cast anonymously at a House Meeting held for that purpose. The Election Committee shall be responsible for collecting and tabulating ballots. The election shall determine a majority winner through successive run offs, and the Election Committee shall be responsible for determining how many candidates to eliminate after each round.


Proxy votes shall be submitted to a member of the Election Committee in written form prior to the election. The Election Committee shall be responsible for entering proxy votes during the election.


The Election Committee shall hear the grievance of any Full House Member from the time of the election to 36 hours post. They shall have closed meetings to determine the legitimacy of the grievance and shall announce any solution to the House. If a solution cannot be reached by the Election Committee, then the election shall be rerun, in part or in full, in accordance with the nature of the grievance. If the grievance is regarding the persons on the Election Committee, then the Upperclassmen Committee shall hear the grievance. They shall have the powers to disband the Election Committee, force a redrawing of a new Election Committee, and begin the election process anew.



\end{document}
