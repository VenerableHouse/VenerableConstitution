\documentclass[10pt]{article} % use larger type; default would be 10pt
\usepackage[utf8]{inputenc} % set input encoding (not needed with XeLaTeX)

\usepackage{geometry} % to change the page dimensions
\geometry{letterpaper}
\usepackage{graphicx} % support the \includegraphics command and options
\setlength\voffset{-0.2in}

\usepackage[dvipsnames]{xcolor} % allows font color change
\usepackage{enumitem} % allows changing enumeration for sublists to match constitution numbering
\usepackage{float}
\usepackage{hyperref}

\title{\includegraphics[width=1.55in]{venerable_crest.png} \\ Constitutional Amendment History}
\author{Venerable House}
\date{\today} % Displays date of latest build

\newcommand{\add}[1]{\textcolor{ForestGreen}{#1}} % \add{new}
\newcommand{\del}[1]{\textcolor{red}{#1}} % \del{old}
\newcommand{\swap}[2]{\del{#1} \add{#2}} % \swap{old}{new}

%%%%%%%%%% Why this document exists %%%%%%%%%%
% The Amendment History document was created as a result of passage of a constitutional amendment written by Elaine Lowinger to keep track of changes to the constitution and the reasoning behind them: 
%    8.4 Records
%    If a proposed amendment is passed, the Constitution is modified to match the amended language. In addition, the proposal for the amendment shall be added to the “Amendment History” document. This includes the date of the amendment's approval, the proposer of the amendment, the reasoning of the amendment, the unamended language, and the approved amended language. The Amendment History document shall be accessible to all Full Members of the House.
% As a result, this document started with four amendments passed on May 26, 2020 (including the Records Amendment) and retroactively includes amendments passed by January 19, 2019. Amendments are in reverse chronological order (newest to oldest).
% It is the duty of the House Secretary to update the constitution and this Amendment History document.


%%%%%%%%%% Structure of Each Entry %%%%%%%%%%
% Each new amendment entry must include:
% 1) the date of the amendment's approval, 
% 2) the proposer of the amendment, 
% 3) the reasoning of the amendment, 
% 4) the existing language, and 
% 5) the approved amended language.
% Note: Use \\ or \newline for line breaks in each bullet point.
%
% Sample Entry:
% \section{Amendment Name}
% \begin{itemize}
% 	\item \textbf{Date Passed:} date
% 	\item \textbf{Author:} name
% 	\item \textbf{Reasoning:} explanation 
% 	\item \textbf{Current Language:} What is currently written in the constitution. Use section numbers and the text, as numbering will shift with further amendments to the constitution.
% 	\item \textbf{Proposed Language:} Changes made from current language, with additions in \textcolor{ForestGreen}{green} and subtractions in \textcolor{red}{red}.
% \end{itemize}


\begin{document}

\maketitle

\section{UCC Transition to ExComm Amendment}
\begin{itemize}
    \item \textbf{Date Passed:} 11/16/2025
    \item \textbf{Author:} Mario Solis and Aaron Zhou
    
    \item \textbf{Reasoning:}
    
    In this past election, I was inclined to run for an ExComm position like Secretary, Treasurer, or Vice President. I thought that not only was I qualified to run, but I could also provide competition in these uncontested races for the better of the House. However, I was unable to do this without dropping my Alley 1 UCC role due to the clauses below in the Constitution. I did not think that dropping my UCC position and leaving a vacancy in Alley 1 was worth all of the work I’d have to do to run, so I didn’t. I also didn’t think it would look good on my part from the House’s perspective.

    I’m making this proposal to prevent these reasons from being what causes an active UCC from running for an elected position they think they are well-suited for. While UCCs and ExComm members provide relatively separate functions to the House, it could be the case that an active UCC is perfect for an ExComm position, but this clause is the only thing stopping them from running. In my opinion, they should be able to run if they feel they would be a good fit for the job. Nevertheless, this candidate should understand that they would take on the duties of both roles and ensure that they will be able to commit the time for them. As a House, we should trust our UCCs to be capable of judging how much they can commit to the House. Even if people don’t think a particular candidate should take both positions for a term and a half, they can express that opinion through simply voting for a different candidate or NO in the election. If ExComm has this opinion after the candidate has won their election, they wouldn’t reappoint this person as a UCC for the next year.

    To recall an ExComm member, the process must be put to a vote in the whole House, and this vote is organized by the UCC. To recall a UCC, ExComm is in charge of voting for or against the removal. If a UCC is elected to ExComm and there is a recall petition against them for either position, they should not be involved in either the organization by the UCCs or the ExComm vote. This is the reasoning for the last two additions in this amendment.
    
    \item \textbf{Current Language:}
        
        2.3 Qualifications and Duties of Appointed Officers
        
        1. One Upper Class Committee-member shall reside in each Venerable House Alley and two shall reside off-campus, except in the case where no suitable volunteers can be found living off-campus, in which case the Off-Campus Upper Class Committee-members may live on-campus. An Upper Class Committee-member shall be a junior or above during their term of office and have been a Full House Member for at least three academic terms prior to their term of office. They shall be available to counsel any member of the House, particularly the occupants of the alley in which they reside or, in the case of an Off-Campus Upper Class Committee-member, any off-campus member. They shall be responsible for maintaining order and enforcing the rules and policies of the House. They shall not concurrently be an Upper Class Committee-member and hold a position on the Executive Committee.
        
        2.5 Recall of Elected Officers

        2. The procedure shall be as follows:

        \begin{enumerate}[(a)]
            \setcounter{enumi}{2}
            \item Voting shall follow the specified procedures for Constitutional amendment elections.
        \end{enumerate}

        2.6 Recall of Upper Class Committee-members

        2. The procedure shall be as follows:
        
        \begin{enumerate}[(a)]
            \item Petitions, containing reasons and purpose, must be submitted either in writing to or in an interview with the Head Upper Class Committee-member and the President;

            \begin{enumerate}[i.]
                \item If the petition is for the Head Upper Class Committee-member, it may be only the President in the interview, and the Head Upper Class Committee-member will be absent from following procedures; similarly, if petitioned person is the Diversity Representative, they will be absent from following procedures;
            \end{enumerate}
            
            \item A meeting will be held in the presence of the entire Executive Committee, the current Head Upper Class Committee-member, and the Diversity Representative for removal;
        \end{enumerate}

    \item \textbf{Proposed Language:}

        2.3 Qualifications and Duties of Appointed Officers
        
        1. One Upper Class Committee-member shall reside in each Venerable House Alley and two shall reside off-campus, except in the case where no suitable volunteers can be found living off-campus, in which case the Off-Campus Upper Class Committee-members may live on-campus. An Upper Class Committee-member shall be a junior or above during their term of office and have been a Full House Member for at least three academic terms prior to their term of office. They shall be available to counsel any member of the House, particularly the occupants of the alley in which they reside or, in the case of an Off-Campus Upper Class Committee-member, any off-campus member. They shall be responsible for maintaining order and enforcing the rules and policies of the House. They shall not concurrently be an Upper Class Committee-member and hold a position on the Executive Committee \add{with the exception of an active Upper Class Committee-member who has won an elected position and will not sign up for the Upper Class Committee for the year following the election}.
        
        2.5 Recall of Elected Officers

        2. The procedure shall be as follows:

        \begin{enumerate}[(a)]
            \setcounter{enumi}{2}
            \item Voting shall follow the specified procedures for Constitutional amendment elections. \add{If the particular officer is also an Upper Class Committee-member, they will not participate in the administration and tabulation of the vote.}
        \end{enumerate}

        2.6 Recall of Upper Class Committee-members

        2. The procedure shall be as follows:
        
        \begin{enumerate}[(a)]
            \item Petitions, containing reasons and purpose, must be submitted either in writing to or in an interview with the Head Upper Class Committee-member and the President;

            \begin{enumerate}[i.]
                \item If the petition is for the Head Upper Class Committee-member, it may be only the President in the interview, and the Head Upper Class Committee-member will be absent from following procedures; similarly, if petitioned person is the Diversity Representative \add{or a member of the Executive Committee}, they will be absent from following procedures;
            \end{enumerate}
            
            \item A meeting will be held in the presence of the entire Executive Committee, the current Head Upper Class Committee-member, and the Diversity Representative for removal;
        \end{enumerate}
\end{itemize}

\section{Amendment Quorum Amendment}
\begin{itemize}
    \item \textbf{Date Passed:} 11/16/2025
    \item \textbf{Author:} Aaron Zhou
    
    \item \textbf{Reasoning:}
    
    Currently, the 1/3 quorum on amendments runs the risk that any proposed amendment will not be passed simply due to a lack of interest. In the long run, this is detrimental to the ability for the House to drive change within broken systems in the House. As a result, we propose several potential changes to the amendment quorum: a reduction to 1/5 of Full Membership of the House, a reduction to 1/4 of Full Membership of the House, and a complete elimination of the amendment quorum. Under Section 8.2.2 of the Venerable Constitution, these options should all be presented to the House and the most favored option should be implemented.
    
    \item \textbf{Current Language:}
        
        8.2 Necessary Vote

        1. In the case of non-conflicting amendments, an amendment to this Constitution passes if at least 2/3 of the votes cast are in favor of the amendment, when quorum is established as per Section 5.2.

        B Mural Amendment
        
        The President will distribute the submitted material to Full Members. In the 48 hours following distribution of the Proposal, comments (including major grievances) are gathered from Full Members. The comments shall be posted immediately following this period. A 72-hour voting period ensues, after which the mural receives House Approval if it receives a 2/3 majority of non-abstain votes with at least 50\% of the members present voting ``Yes'', when those voting constitute at least 1/3 of the Full Membership of the House.

    \item \textbf{Proposed Language:}

        8.2 Necessary Vote

        1. In the case of non-conflicting amendments, an amendment to this Constitution passes if at least 2/3 of the votes cast are in favor of the amendment, \swap{when quorum is established as per Section 5.2}{where quorum is established with 1/4 of Full Membership of the House}.

        B Mural Amendment
        
        The President will distribute the submitted material to Full Members. In the 48 hours following distribution of the Proposal, comments (including major grievances) are gathered from Full Members. The comments shall be posted immediately following this period. A 72-hour voting period ensues, after which the mural receives House Approval if it receives a 2/3 majority of non-abstain votes with at least 50\% of the members present voting ``Yes'', \swap{when those voting constitute at least 1/3 of the Full Membership of the House}{when those voting constitute at least 1/4 of the Full Membership of the House}.
\end{itemize}

\clearpage
\section{Social Quorum Amendment}
\begin{itemize}
    \item \textbf{Date Passed:} 2/7/2025
    \item \textbf{Author:} Randy Ngo
    \item \textbf{Reasoning:} Recently, it has taken extended periods of time to hit the 1/3 quorum requirement on membership meetings after dinner. Low attendance meetings have primarily reached quorum through waiting for proxy votes, rather than an increase of in-person meeting attendance. By lowering the quorum requirement to 1/4, this would reduce the average waiting period before meetings without significant consequences to how representative the vote is of the house.
    
    \item \textbf{Current Language:} \\
        5.2 Quorum \\
        1. A quorum shall consist of:
        \begin{enumerate}[label=(\alph*)]
            \item 1/3 of the Full Membership of the House.
        \end{enumerate}

    \item \textbf{Proposed Language:} \\
        5.2 Quorum \\
        1. A quorum shall consist of:
        \begin{enumerate}[label=(\alph*)]
            \item \textcolor{ForestGreen}{1/4 of the Full Membership of the House for Social Membership Meetings,}
            \item[\textcolor{red}{(a)} \textcolor{ForestGreen}{(b)}] 1/3 of the Full Membership of the House \textcolor{ForestGreen}{for all other House Meetings}.
        \end{enumerate}
\end{itemize}

\newpage

\section{Membership Application Amendment}
\begin{itemize}
	\item \textbf{Date Passed:} 2/7/2025
    \item \textbf{Author:} Joe Giambrone
    \item \textbf{Reasoning:} The method of applying for a membership is not currently codified in the constitution. This would align with our current process of having guests announce themselves at dinner, while additionally allowing those who are unwilling or unable to do so to apply through an alternative method. Additionally, this will provide more flexibility with when membership meetings will take place, allowing for a quicker process.
    
    \item \textbf{Current Language:} \\
        1.2 Membership \\
        1. The types and qualifications of membership are:
        \begin{enumerate}[label=(\alph*)]
            \item[(b)] a Social Member, who is an undergraduate at the California Institute of Technology and has been voted in as such, following a week of consideration, by a simple majority of non-abstain votes, when quorum is established as per Section 5.2;
        \end{enumerate}

    \item \textbf{Proposed Language:} \\
        1.2 Membership \\
        1. The types and qualifications of membership are:
        \begin{enumerate}[label=(\alph*)]
            \item[(b)] a Social Member, who is an undergraduate at the California Institute of Technology and has been voted in as such, \del{following a week of consideration,} by a simple majority of non-abstain votes, when quorum is established as per Section 5.2;
        \end{enumerate}
        \textcolor{ForestGreen}{4. Those seeking a full or social membership shall announce their intent to apply at a House dinner, or, in extenuating circumstances, may send their request to the President and Secretary, who will then announce it to the House Membership. Voting on a new membership application shall take place at a House meeting within the next available calendar week following the application.}
\end{itemize}

\clearpage
\section{Mural Amendment}
\begin{itemize}
    \item \textbf{Date Passed:} 2/7/2025
    \item \textbf{Author:} Elin Stenmark
    \item \textbf{Reasoning:}
    
    There currently exists no constitutional precedent for the process of altering House spaces by painting murals, including who gets to submit a Mural Proposal, how House Approval is sought, and who is deemed responsible for ensuring compliance with Caltech policy and communication with the appropriate stakeholders.

    The Mural process is a cooperative effort between the House and Caltech's Housing Office, and is governed by Caltech policy. Therefore, this amendment seeks to clarify and codify our internal Mural Proposal process while leaving enough flexibility to adapt to new Caltech policy without having to update the constitution in the event that Caltech policy changes.
    
    Any Venerable House member can become a Project Leader, given that Full Members vote on any proposed mural. Quorum and voting seeks to mirror that of House meetings for Full Membership applications.
    
    Given the asynchronous voting period, in the event that a major grievance is identified, by posting the grievances, all members become aware of the alleged issue and are able to vote again with this information in mind. By limiting the instances of voicing grievances to one posting of grievances, we ensure that the proposed mural is not maliciously prevented from being enacted given that it receives enough votes for House Approval. %TODO update probably?

    
    \item \textbf{Current Language:} \\
        1.3 Duties and Privileges \\
        1. The duties and privileges of Full Members apply immediately upon attaining such status.
        These duties and privileges are:
        \begin{enumerate}[label=(\alph*)]
            \item to attend all regular and special House Meetings;
            \item to participate actively in all House functions;
            \item to pay dues;
            \item to have room preference;
            \item to vote on all matters put before the House;
            \item to abide by the rules and regulations made by the Executive Committee, the Upper Class Committee, and the Committee on Student Houses.
        \end{enumerate}
        2. The duties and privileges of Social Members apply immediately upon ratification by The House and are identical to Full Members, with the exception of:
        \begin{enumerate}[label=(\alph*)]
            \item the right to have room preference;
            \item the right to vote on all matters put before The House;
        \end{enumerate}

    \newpage

    \item \textbf{Proposed Language:} \\
        1.3 Duties and Privileges \\
        1. The duties and privileges of Full Members apply immediately upon attaining such status.
        These duties and privileges are:
        \begin{enumerate}[label=(\alph*)]
            \item to attend all regular and special House Meetings;
            \item to participate actively in all House functions;
            \item to pay dues;
            \item to have room preference;
            \item to vote on all matters put before the House; \textcolor{ForestGreen}{\item to propose House murals in accordance with the Mural Amendment (Amendment B);}
            \item[\textcolor{red}{(f)} \textcolor{ForestGreen}{(g)}] to abide by the rules and regulations made by the Executive Committee, the Upper Class Committee, and the Committee on Student Houses.
        \end{enumerate}
        2. The duties and privileges of Social Members apply immediately upon ratification by The House and are identical to Full Members, with the exception of:
        \begin{enumerate}[label=(\alph*)]
            \item the right to have room preference;
            \item the right to vote on all matters put before \swap{The}{the} House;
        \end{enumerate}

        \color{ForestGreen} B Mural Amendment
        
        The designated artist lead, hereby ``Project Leader'', is to notify the President of intent to paint a mural. The Project Leader will be asked to submit a Mural Proposal to the President, containing a mockup design and other pertinent details as specified by the President. If policy violations are identified, the Project Leader will be asked to revise their submission. The Diversity Representative shall be consulted to confirm that the mural is absent of offensive depictions.
        
        The President will distribute the submitted material to Full Members. In the 48 hours following distribution of the Proposal, comments (including major grievances) are gathered from Full Members. The comments shall be posted immediately following this period. A 72-hour voting period ensues, after which the mural receives House Approval if it receives a 2/3 majority of non-abstain votes with at least 50\% of the members present voting ``Yes'', when those voting constitute at least 1/3 of the Full Membership of the House.
        
        A Mural Proposal must receive House Approval before being submitted to Caltech's Housing Office. The Project Leader and President jointly assume responsibility for ensuring that the mural complies with all regulations established by Caltech's Housing Office. Further, the President is responsible for submitting the Mural Proposal to Caltech's Housing Office following House Approval.
\end{itemize}

\newpage

\section{Election Committee Appointment Amendment}
\begin{itemize}
    \item \textbf{Date Passed:} 2/7/2025
    \item \textbf{Author:} Joe Giambrone
    \item \textbf{Reasoning:} Currently, the procedure regarding the actual appointment the Election Committee is not specified. This amendment ensures that it is appointed by the President and the Head Upper Class Committee-member, and clarifies what happens if the President is running.
    
    \item \textbf{Current Language:} \\
        A Procedural Amendment \\
        An Election Committee shall be appointed at the nomination meeting. The Election Committee shall be composed of nine Full House Members: 3 sophomores, 3 juniors, and 3 seniors or above, who shall be selected from a pool of volunteers who are not running for an elected office. In the event that too few members of a specific class volunteer, the remaining members shall be drawn from the other classes.

    \item \textbf{Proposed Language:} \\
        A Procedural Amendment \\
        An Election Committee shall be appointed at the nomination meeting. The Election Committee shall be composed of nine Full House Members: 3 sophomores, 3 juniors, and 3 seniors or above, who shall be \swap{selected}{appointed by the President and Head Upper Class Committee-member} from a pool of volunteers who are not running for an elected office. \add{If the President is running for an elected office, then the members shall be appointed by the Head Upper Class Committee-member alone.} In the event that too few members of a specific class volunteer, the  remaining members shall be drawn from the other classes.
\end{itemize}


\section{Soc/Ath Team Voting Amendment}
\begin{itemize}
    \item \textbf{Date Passed:} 1/24/2025
    \item \textbf{Author:} Randy Ngo
    \item \textbf{Reasoning:}
    
    This amendment clarifies how Social and Athletic team elections are run.

    In the past, full members were allowed to run individually or in a team with other members. This caused two issues. First, individuals were put at a disadvantage if they did not have running mates but had to run against a team of candidates. Secondly, there is currently no mathematical method specified as to how a winner is chosen under an election where both individuals and teams can run. This amendment serves to fix both of those problems.

    To standardize all multi-winner election procedures, the upperclass BoC election process has been modified to be in line with the new Social and Athletic team election procedures. The single transferable vote method\footnote{A more detailed explanation can be found here: \href{https://youtu.be/l8XOZJkozfI}{https://youtu.be/l8XOZJkozfI}} (Droop quota variation\footnote{\href{https://youtu.be/wRc630BSTIg}{https://youtu.be/wRc630BSTIg}}) was chosen as it is generally regarded as the most fair multi-winner electoral system and is commonly used in parliamentary elections around the world.
    \item \textbf{Current Language:}
    
    A Procedural Amendment

    Paper ballots shall be cast anonymously at a House Meeting held for that purpose. The Election Committee shall be responsible for collecting and tabulating ballots. The winner of each election, other than for the BoC representatives, shall be determined as follows: all candidates that do not achieve a majority over ``NO'' are eliminated, and the ranked pairs (Tideman) method is applied to the remaining candidates. The Election Committee shall be responsible for breaking any ties that occur during this process.
    
    For the election of the BoC representatives, each voter may cast two first choice votes and rank their replacements in the case that one of their first choices is eliminated. Votes shall be counted via elimination of the candidate with the lowest number of first choice votes until two candidates receive over 1/3 of the votes each.
    
    \item \textbf{Proposed Language:}
    
    A Procedural Amendment
    
    Paper ballots shall be cast anonymously at a House Meeting held for that purpose. The Election Committee shall be responsible for collecting and tabulating ballots. The winner of each election\del{, other than for the BoC representatives,} shall be determined as follows:

    \setlength{\leftskip}{5mm}
    \add{For positions where multiple individuals are elected, candidates must run individually. All candidates that do not achieve a majority over ``NO'' are eliminated, and the single transferable vote method is applied using the Droop quota.}
    
    \add{For all other positions,} all candidates that do not achieve a majority over ``NO'' are eliminated, and the ranked pairs (Tideman) method is applied to the remaining candidates.
    
    \setlength{\leftskip}{0mm}
    The Election Committee shall be responsible for breaking any ties that occur during \del{this process} \add{these processes}.

    \del{For the election of the BoC representatives, each voter may cast two first choice votes and rank their replacements in the case that one of their first choices is eliminated. Votes shall be counted via elimination of the candidate with the lowest number of first choice votes until two candidates receive over 1/3 of the votes each.}
\end{itemize}

\section{UCC Eligibility Amendment}
\begin{itemize}
    \item \textbf{Date Passed:} 1/24/2025
    \item \textbf{Author:} Randy Ngo
    \item \textbf{Reasoning:} The constitution currently does not allow transfer students to apply to become UCCs until their second year at Caltech, even if they were to be juniors or seniors during their term in office. This amendment serves to expand UCC application eligibility for that scenario.
    
    \item \textbf{Current Language:} \\
        2.3 Duties and Qualifications of Appointed Officers \\
        1. One Upper Class Committee-member shall reside in each Venerable House Alley and two shall reside off-campus, except in the case where no suitable volunteers can be found living off-campus, in which case the Off-Campus Upper Class Committee-members may live on-campus. An Upper Class Committee-member shall have been both enrolled at Caltech for at least five academic terms and a Full House Member for at least three academic terms prior to their term of office. A Peer Advocate shall have been both enrolled at Caltech for at least five academic terms and a Full House Member for at least three academic terms prior to their term of office. They shall be available to counsel any member of the House, particularly the occupants of the alley in which they reside or, in the case of an Off-Campus Upper Class Committee-member, any off-campus member. They shall be responsible for maintaining order and enforcing the rules and policies of the House. They shall not concurrently be an Upper Class Committee-member and hold a position on the Executive Committee.

    \item \textbf{Proposed Language:} \\
        2.3 Duties and Qualifications of Appointed Officers \\
        1. One Upper Class Committee-member shall reside in each Venerable House Alley and two shall reside off-campus, except in the case where no suitable volunteers can be found living off-campus, in which case the Off-Campus Upper Class Committee-members may live on-campus. An Upper Class Committee-member shall \add{be a junior or above during their term of office and} have been \del{both enrolled at Caltech for at least five academic terms and} a Full House Member for at least three academic terms prior to their term of office. A Peer Advocate shall have been both enrolled at Caltech for at least five academic terms and a Full House Member for at least three academic terms prior to their term of office.\footnote{This sentence is subject to removal from the Appointed Positions Current Practices Amendment; however, these two amendments are non-conflicting and thus are not subject to clause 8.2.2.} They shall be available to counsel any member of the House, particularly the occupants of the alley in which they reside or, in the case of an Off-Campus Upper Class Committee-member, any off-campus member. They shall be responsible for maintaining order and enforcing the rules and policies of the House. They shall not concurrently be an Upper Class Committee-member and hold a position on the Executive Committee.
\end{itemize}

\section{Appointed Positions Current Practices Amendment}
\begin{itemize}
    \item \textbf{Date Passed:} 1/24/2025
    \item \textbf{Author:} Noah Howell and Joe Giambrone
    \item \textbf{Reasoning:} Clarified many aspects of appointed positions to align with current practices. There are no ``new'' additions; everything below has already been in effect for a while, sometimes years.
    \begin{itemize}
        \item Fixed title to match previous section on Elected Officers.
        \item Removed almost all mentions of Peer Advocates as they are no longer a House-appointed position, leading to confusion with UCCs.
        \item Replaced ``Diversity Council'' with a single ``Diversity Representative'' as there is only one of them, and updated the name of the Caltech Center for Inclusion and Diversity.
        \item Moved the selection of the Diversity Representative to be during the second round of appointed positions, to keep in line with recent practices.
        \item Updated the role of the Head UCC and Diversity Representative in the selection of appointed positions.
        \item Removed outdated duties of the Historians as they no longer make a scrapbook or maintain a House roll.
        \item Pluralized ``Head Waiters'' as there is more than one of them.
        \item Updated position names of BFD Editors, O'Domhnaills Representatives, IMSS Representatives, BoC Representatives, Head Waiters, and Historians to remove the preceding ``House(-elected)'' or otherwise fix.
        \item Added name and description of the BFD Advisor.
        \item Updated timeline for sign-ups and appointments.
        \item Removed amendment election tabulation from being a UCC responsibility to match recent practices.
        % \item Included requirement that officers with guaranteed rooms must be officers of Venerable House specifically (and not of another house).
        \item Fixed various small typos.
    \end{itemize}

    \item \textbf{Current Language:} \\
        2.3 Duties and Qualifications of Appointed Officers \\
        1. One Upper Class Committee-member shall reside in each Venerable House Alley and two shall reside off-campus\footnote{We denote an off-campus member of Venerable House as a member who does not physically reside in the house.}, except in the case where no suitable volunteers can be found living off-campus, in which case the Off-Campus Upper Class Committee-members may live on-campus. An Upper Class Committee-member shall have been both enrolled at Caltech for at least five academic terms and a Full House Member for at least three academic terms prior to their term of office. A Peer Advocate shall have been both enrolled at Caltech for at least five academic terms and a Full House Member for at least three academic terms prior to their term of office. They shall be available to counsel any member of the House, particularly the occupants of the alley in which they reside or, in the case of an Off-Campus Upper Class Committee-member, any off-campus member. They shall be responsible for maintaining order and enforcing the rules and policies of the House. They shall not concurrently be an Upper Class Committee-member and hold a position on the Executive Committee. \\
        2. The Head Upper Class Committee-member shall have all the qualifications and duties of an Upper Class Committee-member. They shall be the head and ex officio member of the Upper Class Committee. They shall be responsible for the entire frosh selection except for the duties delegated to the President. The Head Upper Class Committee-member shall have previously served as a Venerable Upper Class Committee-member, except in the case where no suitable volunteers can be found. \\
        3. The Diversity Councilmembers shall be responsible for liaising with the Caltech Center for Diversity. They shall also represent the House on the campus-wide Advocacy Committee. They shall be available to advise any member of the House on issues pertaining to diversity. They shall select one Representative to be present during the selection of the Upper Class Committee and the Diversity Council. This Representative may not reapply for a position on the succeeding Diversity Council. In the case where no suitable volunteers can be found, the Head Upper Class Committee-person shall act as a proxy for the Diversity Council during the selection of the Upper Class Committee and the Diversity Council. \\
        4. The Historians shall be responsible for all recording and preservation of House records and traditions, including House scrapbook, House history, and House roll, and for submission of the House article in the Institute yearbook. \\
        5. The Head Waiter shall be responsible for maintaining proper decorum and conduct in the dining room, and shall direct the student waiters. The Head Waiter shall be the Food Representative for Venerable House. \\
        6. The House BFD Editors shall be responsible for publishing the House paper. \\
        7. The House O'Domhnaill's Suppliers shall maintain the House convenience store. \\
        8. The House IMSS Representatives shall act as liaisons between House members and Information Management Systems and Services for network, application, hardware, and software problems. The IMSS Representatives shall include the following: \\ \\
        2.4 Nominations, Elections, and Appointments \\
        5. Upper Class Committee-member sign-ups, Peer Advocate sign-ups, and Diversity Councilmember sign-ups shall be posted concurrently. All Upper Class Committee-members, Peer Advocates, and Diversity Councilmembers shall be appointed at the same meeting. The entire Executive Committee, the current Head Upper Class Committee-member, and the Diversity Council Representative will be present for selection. Selected Peer Advocates will be members of the Upper Class Committee. The term of service for those selected to be members of the Upper Class Committee will be the academic year directly following their appointment, with the exception of the newly-appointed Head Upper Class Committee-member and Diversity Council-members, who will begin serving in their roles during Third Term. \\
        6. The interview committee committee for Peer Advocates will be composed of the Head Upper Class Committee-member, the President, the Sophomore Representative, and the highest-ranking sophomore member of the Executive Committee. If there are no sophomore members of the Executive Committee, then the Vice President will fulfill this role. The interview committee for non-Peer Advocate Upper Class Committee-member positions will be composed of members of the Executive Committee and the Upper Class Committee as the Head Upper Class Committee-member and President see fit. \\
        7. Sign-up lists for non-Upper Class Committee-member appointed offices shall be posted between the first day of Third Term and the third Friday of Third Term. Sign-up lists shall be removed and appointments made by the Executive Committee two weeks thereafter. The term of service for appointed offices shall be the academic year directly following their appointment by the Executive Committee. \\ \\
        2.6 Recall of Upper Class Committee-members \\
        2. The procedure shall be as follows: \\
        (a) Petitions, containing reasons and purpose, must be submitted either in writing to or in an interview with the Head Upper Class Committee-member and the President; \\
        i. If the petition is for the Head Upper Class Committee-member, it may be only the President in the interview, and the Head Upper Class Committee-member will be absent from following procedures; similarly, if petitioned person is the Diversity Council Representative, they will be absent from following procedures; \\
        (b) A meeting will be held in the presence of the entire Executive Committee, the current Head Upper Class Committee-member, and the Diversity Council Representative for removal; \\ \\
        4.3 Powers and Duties of the Committee \\
        The Upper Class Committee: \\
        2. shall be responsible for the administration and tabulation of recall and amendment elections. \\
        3. shall be autonomous in all matter of judicial policy not otherwise specified in this Constitution. \\ \\
        7.2 Guarantees \\
        7. Entire rooms shall be guaranteed to the following officers: \\
        (a) In order of preference, President, Vice President, Secretary, Treasurer, Librarian, Social Team including the Social Manager and Social Mini (up to three members), Athletic Team including the Athletic Manager (up to two members), Head UCC, On Campus UCCs (one member per alley), Peer Advocates, IHC Chair, ASCIT President, BoC Chair, House-elected BoC Representatives (up to two members, with one of the two members being a rising sophomore), O'Domhnaill's Suppliers (up to two members), House Head Waiter (one member), House BFD editors (one member), House Historians (one member) for the entire academic year following the election or appointment provided the officer completes their term. If an officer quits their office mid-term, the room guarantee shall transfer to the succeeding officeholder for the remainder of the guaranteed term. \\
        8. The officer room pick order shall be President, Vice President, Secretary, Treasurer, Librarian, Social Manager, Athletic Manager, Head UCC, On Campus UCCs, Social Mini, IHC Chair, ASCIT President, BoC Chair, Social Team (up to three members including the Social Manager and Social Mini), Athletic Team (up to two members including the Athletic Manager), House-elected BoC Representatives (up to two members, with one of the two members being a rising sophomore), O'Domhnaill's Suppliers (up to two members), House Head Waiter (one member), House BFD editors (one member), House Historians (one member).

    \item \textbf{Proposed Language:} \\
        2.3 \del{Duties and} Qualifications \add{and Duties} of Appointed Officers \\
        1. One Upper Class Committee-member shall reside in each Venerable House Alley and two shall reside off-campus, except in the case where no suitable volunteers can be found living off-campus, in which case the Off-Campus Upper Class Committee-members may live on-campus. An Upper Class Committee-member shall have been both enrolled at Caltech for at least five academic terms and a Full House Member for at least three academic terms prior to their term of office.\footnote{subject to change from the UCC Eligibility Amendment} \del{A Peer Advocate shall have been both enrolled at Caltech for at least five academic terms and a Full House Member for at least three academic terms prior to their term of office.} They shall be available to counsel any member of the House, particularly the occupants of the alley in which they reside or, in the case of an Off-Campus Upper Class Committee-member, any off-campus member. They shall be responsible for maintaining order and enforcing the rules and policies of the House. They shall not concurrently be an Upper Class Committee-member and hold a position on the Executive Committee. \\
        2. The Head Upper Class Committee-member shall have all the qualifications and duties of an Upper Class Committee-member. They shall be the head and ex officio member of the Upper Class Committee. \add{They shall be present during the selection of all appointed offices.} They shall be responsible for the entire frosh selection except for the duties delegated to the President. The Head Upper Class Committee-member shall have previously served as a Venerable Upper Class Committee-member, except in the case where no suitable volunteers can be found. \\
		3. The Diversity \swap{Councilmembers}{Representative} shall be responsible for liaising with the Caltech Center for \add{Inclusion and} Diversity. They shall also represent the House on the campus-wide Advocacy Committee. They shall be available to advise any member of the House on issues pertaining to diversity. They shall \del{select one Representative to} be present during the selection of \swap{the Upper Class Committee and the Diversity Council. This}{all appointed offices. The Diversity} Representative may not reapply for \swap{a position on the succeeding Diversity Council}{this office}. In the \swap{case where no suitable volunteers can be found}{event that the Diversity Representative is absent}, the Head Upper Class Committee-person shall act as a proxy for the Diversity \swap{Council}{Representative} during the selection of \swap{the Upper Class Committee and the Diversity Council}{appointed offices}. \\
        4. The Historians shall be responsible for all recording and preservation of House records\add{,} \del{and} traditions, \del{including House scrapbook, House} \add{and} history, \del{and House roll,} and for submission of the House article in the Institute yearbook. \\
        5. The Head Waiter\add{s} shall be responsible for maintaining proper decorum and conduct in the dining room, and shall direct the student waiters. \swap{The}{One among the} Head Waiter\add{s} shall be the Food Representative for Venerable House. \\
        6. The \del{House} BFD Editors shall be responsible for publishing the House paper. \add{The BFD Advisor shall approve the content of and handle complaints about the House paper.} \\
        7. The \del{House} O'Domhnaill\del{'}s \swap{Suppliers}{Representatives} shall maintain the House convenience store. \\
        8. The \del{House} IMSS Representatives shall act as liaisons between House members and Information Management Systems and Services for network, application, hardware, and software problems. The IMSS Representatives shall include the following: \\ \\
        2.4 Nominations, Elections, and Appointments \\
        5. Upper Class Committee-member sign-ups\del{, Peer Advocate sign-ups, and Diversity Councilmember sign-ups} shall be posted concurrently. All Upper Class Committee-members\del{, Peer Advocates, and Diversity Councilmembers} shall be appointed at the same meeting\add{, which shall take place before the installation of the newly elected officers}. The entire Executive Committee, the current Head Upper Class Committee-member, and the Diversity \del{Council} Representative will be present for selection\add{, with the exception of those who have applied to become a member of the succeeding Upper Class Committee}. \del{Selected Peer Advocates will be members of the Upper Class Committee.} The term of service for those selected to be members of the Upper Class Committee will be the academic year directly following their appointment, with the exception of the newly-appointed Head Upper Class Committee-member \del{and Diversity Council-members}, who will begin serving in their role\swap{s during Third Term}{at the beginning of third term}. \\
        6. \del{The interview committee committee for Peer Advocates will be composed of the Head Upper Class Committee-member, the President, the Sophomore Representative, and the highest-ranking sophomore member of the Executive Committee. If there are no sophomore members of the Executive Committee, then the Vice President will fulfill this role.} The interview committee for \del{non-Peer Advocate} Upper Class Committee-member positions will be composed of members of the Executive Committee and the Upper Class Committee as the Head Upper Class Committee-member and President see fit. \\
        7. Sign-up lists for non-Upper Class Committee-member appointed offices shall be posted \swap{between the first day of Third Term and the third Friday of Third Term. Sign-up lists shall be removed and appointments made by the Executive Committee}{no later than the third Friday of third term, and shall close} two weeks thereafter. \add{All non-Upper Class Committee-member appointed offices shall be appointed at the same meeting, which shall take place within the two weeks following. The entire Executive Committee, the current Head Upper Class Committee-member, and the Diversity Representative shall be present for selection. In the case where a member of the selection committee has applied for an office appointed at this meeting, they shall be absent during the selection of that office.} The term of service for appointed offices shall be the academic year directly following their appointment by the Executive Committee\add{, with the exception of the Diversity Representative, who will begin serving in their role immediately after their appointment}. \\ \\
        2.6 Recall of Upper Class Committee-members \\
        2. The procedure shall be as follows: \\
        (a) Petitions, containing reasons and purpose, must be submitted either in writing to or in an interview with the Head Upper Class Committee-member and the President; \\
        i. If the petition is for the Head Upper Class Committee-member, it may be only the President in the interview, and the Head Upper Class Committee-member will be absent from following procedures; similarly, if petitioned person is the Diversity \del{Council} Representative, they will be absent from following procedures; \\
        (b) A meeting will be held in the presence of the entire Executive Committee, the current Head Upper Class Committee-member, and the Diversity \del{Council} Representative for removal; \\ \\
        4.3 Powers and Duties of the Committee \\
        The Upper Class Committee: \\
        2. shall be responsible for the administration and tabulation of recall \del{and amendment} elections. \\
        3. shall be autonomous in all matter\add{s} of judicial policy not otherwise specified in this Constitution. \\ \\
        7.2 Guarantees \\
        7. Entire rooms shall be guaranteed to the following officers: \\
        (a) In order of preference, President, Vice President, Secretary, Treasurer, Librarian, Social Team including the Social Manager and Social Mini (up to three members), Athletic Team including the Athletic Manager (up to two members), Head UCC, On Campus UCCs (one member per alley), Peer Advocates, IHC Chair, ASCIT President, BoC Chair, \del{House-elected} BoC Representatives (up to two members, with one of the two members being a rising sophomore), O'Domhnaill\del{'}s \swap{Suppliers}{Representatives} (up to two members), \del{House} Head Waiter\add{s} (one member), \del{House} BFD editors (one member), \del{House} Historians (one member) for the entire academic year following the election or appointment provided the officer completes their term. If an officer quits their office mid-term, the room guarantee shall transfer to the succeeding officeholder for the remainder of the guaranteed term. \\
        8. The officer room pick order shall be President, Vice President, Secretary, Treasurer, Librarian, Social Manager, Athletic Manager, Head UCC, On Campus UCCs, Social Mini, IHC Chair, ASCIT President, BoC Chair, Social Team (up to three members including the Social Manager and Social Mini), Athletic Team (up to two members including the Athletic Manager), \del{House-elected} BoC Representatives (up to two members, with one of the two members being a rising sophomore), O'Domhnaill\del{'}s \swap{Suppliers}{Representatives} (up to two members), \del{House} Head Waiter\add{s} (one member), \del{House} BFD editors (one member), \del{House} Historians (one member).
\end{itemize}

\section{Elected Positions Current Practices Amendment}
\begin{itemize}
    \item \textbf{Date Passed:} 1/24/2025
    \item \textbf{Author:} Joe Giambrone
    \item \textbf{Reasoning:} Clarified many aspects of elected positions to align with current practices. There are no ``new'' additions; everything below has already been in effect for a while, sometimes years.
    \begin{itemize}
        \item Updated the duties of the Vice President, Secretary, Treasurer, Librarian, Social Team, Athletic Team, and Sophomore Representative, as their current descriptions are lacking or outdated.
        \item Updated the timeline for nominations.
        \item Updated the requirements for President, Vice President, BoC Representative, and Electcomm to allow for super-seniors.
        \item Removed the requirement for Electcomm to be fully random to allow for more flexibility in the selection process, such as first-come, first-serve.
    \end{itemize}

    \item \textbf{Current Language:} \\
        2.2 Qualifications and Duties of Elected Officers \\
        1. The President shall reside in Venerable House and shall be a junior or senior for the latter portion of their term in office. They shall represent the House on the Interhouse Committee, preside over House Meetings, be chairperson of the Executive Committee, and be an ex-officio member of all House committees, and shall be responsible for selecting Full Members to attend frosh picks. \\
        2. The Vice President shall be a junior or senior for the latter portion of their term in office. They shall assume the duties of the President in the event of the President's absence. They shall be responsible for maintaining the external relations of the House, for bringing guests to the House, and shall act as official host for all House guests. They shall be the Venerable representative on the Review Committee. They shall share duties with the Treasurer in maintaining house property. \\
        3. The Secretary shall record minutes, including all motions made and the votes on these by all officers present, of all House, Executive Committee, and Upper Class Committee meetings; and post publicly a copy of the minutes of all open meetings. They shall have custody of all House records not specifically delegated to the custody of another officer, and in general shall perform all the duties of a corresponding and recording secretary. They shall be responsible for room assignments. \\
        4. The Treasurer shall be responsible to the House for all receipts and expenditures, shall maintain an adequate book-keeping system, shall submit a report to the Executive Committee at each regularly scheduled meeting, shall propose a budget to the Executive Committee for each term, and shall submit to the House a written report of the finances at the end of each term. They shall share duties with the Vice President in maintaining house property. \\
        5. The Librarian shall reside in Venerable House and shall be a sophomore for the latter portion of their term in office. They shall be responsible for procuring magazines and newspapers, and for maintaining the House Library, including its records. They shall publish a House list at the beginning of each term. They shall be responsible for frosh orientation and organizing the frosh members of the House for the performance of those tasks assigned to them. \\
        6. The office of Social Team shall be held by not more than five members of the House, among them a Social Manager and Social Mini. The Social Team will be responsible for organizing social functions, and for publishing a calendar of such events. \\
        7. The office of Athletic Team shall be held by not more than three members of the House, among them a single Athletic Manager. The Athletic Team shall be responsible for all athletic equipment used by the House, for the participation of the House in Interhouse and intrahouse athletics, and in general shall organize and promote the athletic activities of the House. \\
        8. The office of Sophomore Representative shall be held by a sophomore for the latter portion of their term in office. They shall be responsible for representing the thoughts and ideas of the sophomore members of the House on the Executive Committee. \\ \\
        2.4 Nominations, Elections, and Appointments \\
        1. Nominations for ARC representative, CRC representative, three BoC representatives, Sophomore Representative, Athletic Manager, Athletic Team, Social Manager, Social Mini, Social Team, Librarian, Treasurer, Secretary, Vice President, and President, in that order, shall be made at a meeting of the House held for that purpose between the beginning of the second term and the third Friday of second term. Nominations shall remain open until 36 hours before the start of the elections as described in section 2.4.2. \\
        (a) Two of the three BoC representatives shall be a junior or senior for the latter portion of their term in office. One of the three BoC representatives shall be a sophomore for the latter portion of their term in office.
        \\ \\
        A Procedural Amendment \\
        An Election Committee shall be appointed at the nomination meeting. The Election Committee shall be composed of nine Full House Members: 3 sophomores, 3 juniors and 3 seniors, who shall be randomly selected from a pool of volunteers who are not running for an elected office. In the event that too few members of a specific class volunteer, the remaining members shall be drawn from the other classes.

    \item \textbf{Proposed Language:} \\
        2.2 Qualifications and Duties of Elected Officers \\
        1. The President shall reside in Venerable House and shall be a junior or \swap{senior}{above} for the latter portion of their term in office. They shall represent the House on the Interhouse Committee, preside over House Meetings, be chairperson of the Executive Committee, and be an ex-officio member of all House committees, and shall be responsible for selecting Full Members to attend frosh picks. \\
        2. The Vice President shall be a junior or \swap{senior}{above} for the latter portion of their term in office. They shall assume the duties of the President in the event of the President's absence. They shall be responsible for maintaining the external relations of the House, for bringing guests to the House, and shall act as official host for all House guests. They shall be the Venerable representative on the Review Committee \add{and the Caltech Y}. They shall share duties with the Treasurer in maintaining house property. \\
        3. The Secretary shall record minutes\del{, including all motions made and the votes on these by all officers present,} of all House\swap{,}{ and} Executive Committee\del{, and Upper Class Committee} meetings; and post publicly a copy of the minutes of all open meetings. \add{They shall maintain House communication channels and alumni relations. They shall be responsible for room assignments and serve as the House Picks Officer.} They shall have custody of \add{the House Membership list, the House Calendar, the House Constitution, and} all House records not specifically delegated to \del{the custody of} another officer, and in general shall perform all the duties of a corresponding and recording secretary. \del{They shall be responsible for room assignments.} \\
        4. The Treasurer shall be responsible to the House for all receipts and expenditures\swap{,}{\!\!. They} shall maintain an adequate book-keeping system, \del{shall submit a report to the Executive Committee at each regularly scheduled meeting,} shall propose a budget to the Executive Committee for each \swap{term}{academic year}, and shall submit to the House a written report of the finances at the end of each term. \add{They shall be responsible for the procurement and distribution of House-owned subscription services.} They shall share duties with the Vice President in maintaining house property. \\
        5. The Librarian shall reside in Venerable House and shall be a sophomore for the latter portion of their term in office. They shall be responsible \del{for procuring magazines and newspapers, and} for maintaining the House Library, including its records. \del{They shall publish a House list at the beginning of each term.} They shall be responsible for frosh orientation\add{,} \del{and} organizing the frosh members of the House for the performance of those tasks assigned to them\add{, and organizing frosh-oriented activities}. \add{They shall maintain a list of the birthdays of House Members and announce them accordingly.} \\
        6. The office of Social Team shall be held by not more than five members of the House, among them a Social Manager and Social Mini. The Social Team will be responsible for organizing social functions\swap{, and for publishing a calendar of such events}{and House trips}. \\
        7. The office of Athletic Team shall be held by not more than three members of the House, among them a single Athletic Manager. The Athletic Team shall be responsible for all athletic equipment used by the House, for the participation of the House in Interhouse and intrahouse athletics, \add{for the procurement, sale, and distribution of House merchandise to House Members and alumni,} and in general shall organize and promote the athletic activities of the House. \\
        8. The office of Sophomore Representative shall be held by a sophomore for the latter portion of their term in office. \add{They shall be responsible for organizing events for the frosh and the sophomores.} They shall be responsible for representing the thoughts and ideas of the sophomore members of the House on the Executive Committee. \\ \\
        2.4 Nominations, Elections, and Appointments \\
        1. Nominations for ARC representative, CRC representative, three BoC representatives, Sophomore Representative, Athletic Manager, Athletic Team, Social Manager, Social Mini, Social Team, Librarian, Treasurer, Secretary, Vice President, and President, in that order, shall be made at a meeting of the House held for that purpose between the beginning of the second term and the \swap{third Friday}{fifth Sunday} of second term. Nominations shall remain open until 36 hours before the start of the elections as described in \swap{section}{Section} 2.4.2. \\
        (a) Two of the three BoC representatives shall be a junior or \swap{senior}{above} for the latter portion of their term in office. One of the three BoC representatives shall be a sophomore for the latter portion of their term in office.
        \\ \\
        A Procedural Amendment \\
        An Election Committee shall be appointed at the nomination meeting. The Election Committee shall be composed of nine Full House Members: 3 sophomores, 3 juniors\add{,} and 3 seniors \add{or above}, who shall be \del{randomly} selected from a pool of volunteers who are not running for an elected office. In the event that too few members of a specific class volunteer, the remaining members shall be drawn from the other classes.

\end{itemize}

\section{Quorum Current Practices Amendment}
\begin{itemize}
    \item \textbf{Date Passed:} 1/24/2025
    \item \textbf{Author:} Joe Giambrone and Noah Howell 
    \item \textbf{Reasoning:} Clarified many aspects of quorum to align with current practices. There are no ``new'' additions; everything below has already been in effect for a while, sometimes years.
    \begin{itemize}
        \item Replaced instances of ``elected'' with ``voted in'' or ``voted on'' to better reflect the process of gaining membership and remove ambiguity with elected positions.
        \item Clarified the requirements to become a Social Member, as previously ``abstain'' counted as ``no''.
        \item Clarified the general quorum requirement for House meetings (including membership meetings, amendments meetings, and election meetings), and added references to the section containing this requirement where applicable, to remove ambiguity.
        \item Updated membership type of the Resident Associates.
        \item Removed the public ``roll of those members who have voted'' for amendments, as this violated the secrecy of the ballot and is no longer practiced.
    \end{itemize}
    
    \item \textbf{Current Language:} \\
        1.2 Membership \\
        1. The types and qualifications of membership are: \\
        (a) a Full Member, who is an undergraduate of the California Institute of Technology and is picked as such at the end of Rotation Week, or is elected at a House meeting by a 2/3 majority of non-abstain votes with at least 50\% of the members present voting ``Yes''; \\
        (b) a Social Member, who is an undergraduate at the California Institute of Technology and has been elected as such, following a week of consideration, by a simple majority of those voting, when those voting constitute at least 1/3 of the House Membership; \\
        (d) the Resident Associates, who shall be considered to be non-voting Full Members. \\
        2. Full and social members who discontinue paying their house dues while they are registered students at the California Institute of Technology must be re-elected at a house meeting as provided for in Section 1.2, part 1a. \\ \\
        5.2 Quorum \\
        1. A quorum shall consist of the larger of: \\
        (a) 1/3 of the Full-Membership of the House, or \\
        (b) 80\% of the average attendance of House meetings held during the three hundred sixty-five (365) days prior to the announced date of the meeting regardless of whether those meetings established quorum. \\ \\
        8.2 Necessary Vote \\
        1. In the case of non-conflicting amendments, an amendment to this Constitution passes if at least 2/3 of the votes cast are in favor of the amendment. \\ \\
        8.3 Procedure \\
        3. Voting \\
        (d) A roll of those members who have voted shall be available throughout the election. \\ \\
        A Procedural Amendment \\
        Paper ballots shall be cast anonymously at a House Meeting held for that purpose. The Election Committee shall be responsible for collecting and tabulating ballots. The winner of each election, other than for the BoC representatives, shall be determined as follows: all candidates that do not achieve a majority over “NO” are eliminated, and the ranked pairs (Tideman) method is applied to the remaining candidates. The Election Committee shall be responsible for breaking any ties that occur during this process.

    \item \textbf{Proposed Language:} \\
        1.2 Membership \\
        1. The types and qualifications of membership are: \\
        (a) a Full Member, who is an undergraduate of the California Institute of Technology and is picked as such at the end of Rotation Week, or is \swap{elected}{voted in} at a House meeting by a 2/3 majority of non-abstain votes with at least 50\% of the members present voting ``Yes''\add{, when quorum is established as per Section 5.2}; \\
        (b) a Social Member, who is an undergraduate at the California Institute of Technology and has been \swap{elected}{voted in} as such, following a week of consideration, by a simple majority of \swap{those voting, when those voting constitute at least 1/3 of the House Membership}{non-abstain votes, when quorum is established as per Section 5.2}; \\
        (d) the Resident Associates, who shall be considered to be \swap{non-voting Full}{Associate} Members. \\
        2. Full and social members who discontinue paying their house dues while they are registered students at the California Institute of Technology must be \swap{re-elected}{voted on again} at a house meeting as provided for in Section 1.2, part 1\swap{a}{to regain their membership}. \\ \\
        5.2 Quorum \\
        1. A quorum shall consist of \del{the larger of}: \\
        (a) 1/3 of the Full\swap{-}{}Membership of the House\add{.} \del{, or \\
        (b) 80\% of the average attendance of House meetings held during the three hundred sixty-five (365) days prior to the announced date of the meeting regardless of whether those meetings established quorum.} \\ \\
        8.2 Necessary Vote \\
        1. In the case of non-conflicting amendments, an amendment to this Constitution passes if at least 2/3 of the votes cast are in favor of the amendment\add{, when quorum is established as per Section 5.2}. \\ \\
        8.3 Procedure \\
        3. Voting\add{:} \\
        \del{(d) A roll of those members who have voted shall be available throughout the election.} \\ \\
        A Procedural Amendment \\
        Paper ballots shall be cast anonymously at a House Meeting held for that purpose. The Election Committee shall be responsible for collecting and tabulating ballots. \add{Quorum must be established as per Section 5.2.} The winner of each election, other than for the BoC representatives, shall be determined as follows: all candidates that do not achieve a majority over “NO” are eliminated, and the ranked pairs (Tideman) method is applied to the remaining candidates. The Election Committee shall be responsible for breaking any ties that occur during this process.
\end{itemize}

\section{UCC Recall Amendment}
\begin{itemize}
	\item \textbf{Date Passed:} 2/16/2024
	\item \textbf{Author:} Emily Choe
	\item \textbf{Reasoning:} As of now, there is no process for the removal of UCCs. To ensure the protection of the program as well as a fair process for the petitioned party, we would like to add a process for the recall of UCCs.
	\item \textbf{Current Language:} \\
	2.5 Recall \\
	1. Any elected officer may be recalled by a majority vote of the voting membership of the House. 2. The procedure shall be as follows: \\
	(a) Petitions, containing reasons and purpose, must be submitted in writing, endorsed by 1/3 of the total Full Membership, at a House meeting held for that purpose; \\
	(b) The petitions must be posted immediately following the presentation and remain posted until voting is closed; \\
	(c) Voting shall follow the specified procedures for Constitutional amendment elections. \\
	3.4 Powers \\
	1. The Executive Committee shall formulate House rules and policies, and assume all duties not otherwise delegated. \\
	2. It shall appoint all offices described under Section 2.3 of this Constitution. \\
	3. It shall designate associate members. \\
	4. It shall have the power of dismissal over all appointed officers, other than Upper Class Committee-members.
	\item \textbf{Proposed Language:} \\
	2.5 Recall \textcolor{ForestGreen}{of Elected Officers} \\
	1. Any elected officer may be recalled by a majority vote of the voting membership of the House. 2. The procedure shall be as follows: \\
	(a) Petitions, containing reasons and purpose, must be submitted in writing, endorsed by 1/3 of the total Full Membership, at a House meeting held for that purpose; \\
	(b) The petitions must be posted immediately following the presentation and remain posted until voting is closed; \\
	(c) Voting shall follow the specified procedures for Constitutional amendment elections. \\
	\textcolor{ForestGreen}{2.6 Recall of Upper Class Committee-members} \\
	\textcolor{ForestGreen}{1. Any appointed Upper Class Committee-member may be recalled by a unanimous vote of the Executive Committee of the House.} \\
	\textcolor{ForestGreen}{2. The procedure shall be as follows:} \\
	\textcolor{ForestGreen}{(a) Petitions, containing reasons and purpose, must be submitted either in writing to or in an interview with the Head Upper Class Committee-member and the President;} \\
	\textcolor{ForestGreen}{i. If the petition is for the Head Upper Class Committee-member, it may be only the President in the interview, and the Head Upper Class Committee-member will be absent from following procedures; similarly, if petitioned person is the Diversity Council Representative, they will be absent from following procedures;} \\
	\textcolor{ForestGreen}{(b) A meeting will be held in the presence of the entire Executive Committee, the current Head Upper Class Committee-member, and the Diversity Council Representative for removal;} \\
	\textcolor{ForestGreen}{(c) The vacancy will be filled as stated in Section 2.4, Paragraph 8.} \\
	3.4 Powers \\
	1. The Executive Committee shall formulate House rules and policies, and assume all duties not otherwise delegated. \\
	2. It shall appoint all offices described under Section 2.3 of this Constitution. \\
	3. It shall designate associate members. \\
	4. It shall have the power of dismissal over all appointed officers\textcolor{red}{, other than Upper Class Committee-members}.
\end{itemize}

\section{Off-Campus Rep Amendment}
\begin{itemize}
	\item \textbf{Date Passed:} 2/2/2024
	\item \textbf{Author:} Randy Ngo
	\item \textbf{Reasoning:} Ever since Bechtel opened in Fall 2018, the importance of the Off-Campus Rep position has significantly reduced. To improve the functionality of Excomm without adjusting its size, this amendment proposes to replace the Off-Campus Rep position with a secondary Soc Man (Social Mini). \\
	Currently, Soc Man is one of the most demanding house positions due to its responsibility of organizing nearly all house events. By adding the Soc Mini position, it would alleviate some of the workload from the Soc Man, while improving continuity in the yearly transition between Social Teams. This Excomm structure of having multiple Social Managers has been well tested in all seven other houses. The current Off-Campus Rep role will be merged with the Soc Mini position, ensuring the voice of off-campus members remains represented on Excomm.
	\item \textbf{Current Language:} \\
	2.2.6. The office of Social Team shall be held by not more than five members of the House, among them a single Social Manager. The Social Team will be responsible for organizing social functions, and for publishing a calendar of such events. \\
	2.2.9. The office of Off-Campus Representative shall be held by a member of the House who has resided off-campus for at least one full academic term. They shall be responsible for representing the thoughts and ideas of the Off-Campus members of the House on the Executive Committee. \\
	2.2.10. The order of succession shall be President, Vice President, Secretary, Treasurer, Librarian, Social Manager, Athletic Manager, Sophomore Representative, and Off-Campus Representative. \\
	2.4.1. Nominations for ARC representative, CRC representative, three BoC representatives, Off-Campus Representative, Sophomore Representative, Athletic Manager, Athletic Team, Social Manager, Social Team, Librarian, Treasurer, Secretary, Vice President, and President, in that order, shall be made at a meeting of the House held for that purpose between the beginning of the second term and the third Friday of second term. Nominations shall remain open until 36 hours before the start of the elections as described in section 2.4.2. \\
	2.4.2. Election of all officers except Social Team and Athletic Team shall be carried out at a House Meeting held for that purpose within one week after the nomination meeting. The order of election shall be in the following order: President, Vice President, Secretary, Treasurer, Librarian, Social Manager, Athletic Manager, Sophomore Representative, Off-Campus Representative, three BoC representatives, CRC representative, ARC representative. Elections shall be managed by an Election Committee in accordance with the Procedural Amendment (Amendment A). \\
	3.2. The Executive Committee shall consist of the President, who shall act as chairperson, the Vice President, Secretary, Treasurer, Librarian, Social Manager, Athletic Manager, Sophomore Representative, and Off-Campus Representative. \\
	7.2.7. Entire rooms shall be guaranteed to the following officers: (a) In order of preference, President, Vice President, Secretary, Treasurer, Librarian, Social Team including the Social Manager (up to three members), Athletic Team including the Athletic Manager (up to two members), Head UCC, On Campus UCCs (one member per alley), Peer Advocates, IHC Chair, ASCIT President, BoC Chair, House-elected BoC Representatives (up to two members, with one of the two members being a rising sophomore), O'Domhnaill's Suppliers (up to two members), House Head Waiter (one member), House BFD editors (one member), House Historians (one member) \\
	7.2.8. The officer room pick order shall be President, Vice President, Secretary, Treasurer, Librarian, Social Manager, Athletic Manager, Head UCC, On Campus UCCs, IHC Chair, ASCIT President, BoC Chair, Social Team, Athletic Team, House-elected BoC Representatives (up to two members, with one of the two members being a rising sophomore), O'Domhnaill's Suppliers (up to two members), House Head Waiter (one member), House BFD editors (one member), House Historians (one member).
	\item \textbf{Proposed Language:} \\
	2.2.6. The office of Social Team shall be held by not more than five members of the House, among them a \textcolor{red}{single} Social Manager \textcolor{ForestGreen}{and Social Mini}. The Social Team will be responsible for organizing social functions, and for publishing a calendar of such events. \\
	\textcolor{red}{2.2.9. The office of Off-Campus Representative shall be held by a member of the House who has resided off-campus for at least one full academic term. They shall be responsible for representing the thoughts and ideas of the Off-Campus members of the House on the Executive Committee.} \\
	2.2.10. The order of succession shall be President, Vice President, Secretary, Treasurer, Librarian, Social Manager, Athletic Manager, \textcolor{ForestGreen}{Social Mini, and} Sophomore Representative\textcolor{red}{, and Off-Campus Representative}. \\
	2.4.1. Nominations for ARC representative, CRC representative, three BoC representatives, \textcolor{red}{Off-Campus Representative,} Sophomore Representative, Athletic Manager, Athletic Team, Social Manager, \textcolor{ForestGreen}{Social Mini,} Social Team, Librarian, Treasurer, Secretary, Vice President, and President, in that order, shall be made at a meeting of the House held for that purpose between the beginning of the second term and the third Friday of second term. Nominations shall remain open until 36 hours before the start of the elections as described in section 2.4.2. \\
	2.4.2. Election of all officers except Social Team and Athletic Team shall be carried out at a House Meeting held for that purpose within one week after the nomination meeting. The order of election shall be in the following order: President, Vice President, Secretary, Treasurer, Librarian, Social Manager, Athletic Manager, \textcolor{ForestGreen}{Social Mini,} Sophomore Representative, \textcolor{red}{Off-Campus Representative,} three BoC representatives, CRC representative, ARC representative. Elections shall be managed by an Election Committee in accordance with the Procedural Amendment (Amendment A). \\
	3.2. The Executive Committee shall consist of the President, who shall act as chairperson, the Vice President, Secretary, Treasurer, Librarian, Social Manager, Athletic Manager, \textcolor{ForestGreen}{Social Mini, and} Sophomore Representative\textcolor{red}{, and Off-Campus Representative}. \\
	7.2.7. Entire rooms shall be guaranteed to the following officers: (a) In order of preference, President, Vice President, Secretary, Treasurer, Librarian, Social Team including the Social Manager \textcolor{ForestGreen}{and Social Mini} (up to three members), Athletic Team including the Athletic Manager (up to two members), Head UCC, On Campus UCCs (one member per alley), Peer Advocates, IHC Chair, ASCIT President, BoC Chair, House-elected BoC Representatives (up to two members, with one of the two members being a rising sophomore), O'Domhnaill's Suppliers (up to two members), House Head Waiter (one member), House BFD editors (one member), House Historians (one member). \\
	7.2.8. The officer room pick order shall be President, Vice President, Secretary, Treasurer, Librarian, Social Manager, \textcolor{ForestGreen}{Social Mini,} Athletic Manager, Head UCC, On Campus UCCs, IHC Chair, ASCIT President, BoC Chair, Social Team \textcolor{ForestGreen}{(up to three members including the Social Manager and Social Mini)}, Athletic Team \textcolor{ForestGreen}{(up to two members including the Athletic Manager)}, House-elected BoC Representatives (up to two members, with one of the two members being a rising sophomore), O'Domhnaill's Suppliers (up to two members), House Head Waiter (one member), House BFD editors (one member), House Historians (one member).
\end{itemize}

\section{BoC Amendment}
\begin{itemize}
	\item \textbf{Date Passed:} 1/29/2024
	\item \textbf{Author:} Emily Choe
	\item \textbf{Reasoning:} As of now, there are only two BoC representatives. However, the amount of BoC reps is not sufficient for the duties of the BoC. Therefore, the IHC was requested to increase the amount of BoC reps in each house, with more upperclassmen representatives with more course experience.
	\item \textbf{Current Language:} \\
	2.4 Nominations, Elections, and Appointments \\
	1. Nominations for ARC representative, CRC representative, two BoC representatives, OffCampus Representative, Sophomore Representative, Athletic Manager, Athletic Team, Social Manager, Social Team, Librarian, Treasurer, Secretary, Vice President, and President, in that order, shall be made at a meeting of the House held for that purpose between the beginning of the second term and the third Friday of second term. Nominations shall remain open until 36 hours before the start of the elections as described in section 2.4.2. \\
	2. Election of all officers except Social Team and Athletic Team shall be carried out at a House Meeting held for that purpose within one week after the nomination meeting. The order of election shall be in the following order: President, Vice President, Secretary, Treasurer, Librarian, Social Manager, Athletic Manager, Sophomore Representative, Off-Campus Representative, two BoC representatives, CRC representative, ARC representative. Elections shall be managed by an Election Committee in accordance with the Procedural Amendment (Amendment A). \\
	\textit{And} \\
	7.2 Guarantees \\
	7. Entire rooms shall be guaranteed to the following officers: \\
	(a) In order of preference, President, Vice President, Secretary, Treasurer, Librarian, Social Team including the Social Manager (up to three members), Athletic Team including the Athletic Manager (up to two members), Head UCC, On Campus UCCs (one member per alley), Peer Advocates, IHC Chair, ASCIT President, BoC Chair, House-elected BoC Representatives (up to two members), O'Domhnaill's Suppliers (up to two members), House Head Waiter (one member), House BFD editors (one member), House Historians (one member) \\
	(b) for the entire academic year following the election or appointment provided the officer completes their term. If an officer quits their office mid-term, the room guarantee shall transfer to the succeeding officeholder for the remainder of the guaranteed term. \\
	8. The officer room pick order shall be President, Vice President, Secretary, Treasurer, Librarian, Social Manager, Athletic Manager, Head UCC, On Campus UCCs, IHC Chair, ASCIT President, BoC Chair, Social Team, Athletic Team, House-elected BoC Representatives (up to two members), O'Domhnaill's Suppliers (up to two members), House Head Waiter (one member), House BFD editors (one member), House Historians (one member).
	\item \textbf{Proposed Language:} \\
	2.4 Nominations, Elections, and Appointments \\
	1. Nominations for ARC representative, CRC representative, \textcolor{red}{two} \textcolor{ForestGreen}{three} BoC representatives, OffCampus Representative, Sophomore Representative, Athletic Manager, Athletic Team, Social Manager, Social Team, Librarian, Treasurer, Secretary, Vice President, and President, in that order, shall be made at a meeting of the House held for that purpose between the beginning of the second term and the third Friday of second term. Nominations shall remain open until 36 hours before the start of the elections as described in section 2.4.2. \\
	\textcolor{ForestGreen}{(a) Two of the three BoC representatives shall be a junior or senior for the latter portion of their term in office. One of the three BoC representatives shall be a sophomore for the latter portion of their term in office.} \\
	2. Election of all officers except Social Team and Athletic Team shall be carried out at a House Meeting held for that purpose within one week after the nomination meeting. The order of election shall be in the following order: President, Vice President, Secretary, Treasurer, Librarian, Social Manager, Athletic Manager, Sophomore Representative, Off-Campus Representative, \textcolor{red}{two} \textcolor{ForestGreen}{three} BoC representatives, CRC representative, ARC representative. Elections shall be managed by an Election Committee in accordance with the Procedural Amendment (Amendment A). \\
	\textit{And} \\
	7.2 Guarantees \\
	7. Entire rooms shall be guaranteed to the following officers: \\
	(a) In order of preference, President, Vice President, Secretary, Treasurer, Librarian, Social Team including the Social Manager (up to three members), Athletic Team including the Athletic Manager (up to two members), Head UCC, On Campus UCCs (one member per alley), Peer Advocates, IHC Chair, ASCIT President, BoC Chair, House-elected BoC Representatives (up to two members\textcolor{ForestGreen}{, with one of the two members being a rising sophomore}), O'Domhnaill's Suppliers (up to two members), House Head Waiter (one member), House BFD editors (one member), House Historians (one member) for the entire academic year following the election or appointment provided the officer completes their term. If an officer quits their office mid-term, the room guarantee shall transfer to the succeeding officeholder for the remainder of the guaranteed term. \\
	8. The officer room pick order shall be President, Vice President, Secretary, Treasurer, Librarian, Social Manager, Athletic Manager, Head UCC, On Campus UCCs, IHC Chair, ASCIT President, BoC Chair, Social Team, Athletic Team, House-elected BoC Representatives (up to two members\textcolor{ForestGreen}{, with one of the two members being a rising sophomore}), O'Domhnaill's Suppliers (up to two members), House Head Waiter (one member), House BFD editors (one member), House Historians (one member).
\end{itemize}

\section{Name Change Amendment}
\begin{itemize}
	\item \textbf{Date Passed:} 2/9/2022
	\item \textbf{Author:} Chase Blanchette
	\item \textbf{Reasoning:} the Constitution should be updated to reflect the new name of the House.
	\item \textbf{Current Language:} \\
	The name ``Ruddock" is used to refer to the House.
	\item \textbf{Proposed Language:} \\
	All instances of the name ``Ruddock" are changed to ``Venerable."
\end{itemize}

\section{AdComm Amendment}
\begin{itemize}
	\item \textbf{Date Passed:} 2/9/2022
	\item \textbf{Author:} Chase Blanchette
	\item \textbf{Reasoning:} AdComm rep is a new position where the Diversity Rep is well suited to represent the House. The current Diversity Rep, Suchitra, is also our AdComm rep.
	\item \textbf{Current Language:} \\
	2.3 Duties and Qualifications of Appointed Officers \\
	3. The Diversity Councilmembers shall be responsible for liaising with the Caltech Center for Diversity. They shall be available to advise any member of the House on issues pertaining to diversity. 
	\item \textbf{Proposed Language:} \\
	2.3 Duties and Qualifications of Appointed Officers \\
	3. The Diversity Councilmembers shall be responsible for liaising with the Caltech Center for Diversity. \textcolor{ForestGreen}{They shall also represent the House on the campus-wide Advocacy Committee. }They shall be available to advise any member of the House on issues pertaining to diversity. 
\end{itemize}

\section{Food Representative Amendment}
\begin{itemize}
	\item \textbf{Date Passed:} 5/26/2020
	\item \textbf{Author:} Sierra Lopezalles
	\item \textbf{Reasoning:} The position of Food Representative is not currently in the constitution, this amendment both adds Food Rep to the constitution and makes it a part of the role of being Head Waiter. The Head Waiter is the person best suited to being Food Rep since they are present at most dinners and are best able to judge the house's response to meals.
	\item \textbf{Current Language:} \\
	2.3 Duties and Qualifications of Appointed Officers \\
	5. The Head Waiter shall be responsible for maintaining proper decorum and conduct in the dining room, and shall direct the student waiters.
	\item \textbf{Proposed Language:} \\
	2.3 Duties and Qualifications of Appointed Officers \\
	5. The Head Waiter shall be responsible for maintaining proper decorum and conduct in the dining room, and shall direct the student waiters. \textcolor{ForestGreen}{The Head Waiter shall be the Food Representative for Ruddock House.}
\end{itemize}

\section{Election Amendment for Online Voting}
\begin{itemize}
	\item \textbf{Date Passed:} 5/26/2020
	\item \textbf{Author:} Sierra Lopezalles
	\item \textbf{Reasoning:} Currently Electcomm uses an online google form for proxy voting. This amendment would align the constitution to the procedure that Electcomm is already using.
	\item \textbf{Current Language:} \\
	A Procedural Amendment \\
	Proxy votes shall be submitted to a member of the Election Committee in written form prior to the election. The Election Committee shall be responsible for entering proxy votes during the election.
	\item \textbf{Proposed Language:} \\
	A Procedural Amendment \\
	Proxy votes shall be submitted to a member of the Election Committee \textcolor{red}{in written form} prior to the election. The Election Committee shall be responsible for entering proxy votes during the election. 
\end{itemize}

\section{Election Amendment for Timeline Modifications}
\begin{itemize}
	\item \textbf{Date Passed:} 5/26/2020
	\item \textbf{Author:} Sierra Lopezalles
	\item \textbf{Reasoning:} \\
	This amendment fixes outdated language regarding the timing of elections. In the past, each position was nominated and elected before the next position could begin. Since we have moved to electing all positions at the same meeting, this text must also be removed in order to align the constitution to our current procedures. \\
	This amendment also adjusts the timeline for electing Soc Team and Ath Team. This ensures that a Soc Man and Ath Man will have been elected before voting on the team commences. Additionally, it extends the voting period to a week, as has been done in the past 2 years. \\
	This amendment also removes text about requiring a majority of votes since this is conflict with the Procedural Amendment. \\
	The Procedural Amendment also requires that ballots be submitted anonymously, thus it does change the procedure to remove the part about secret ballots.
	\item \textbf{Current Language:} \\
	2.3. Duties and Qualifications of Appointed Officers \\
	5. The Head Waiter shall be responsible for maintaining proper decorum and conduct in the dining room, and shall direct the student waiters.2.4 Nominations, Elections, and Appointments \\
	2. Election of all officers except Social Team and Athletic Team shall be carried out at a House Meeting held for that purpose within one week after the nomination meeting. The order of election shall be in the following order: President, Vice President, Secretary, Treasurer, Librarian, Social Manager, Athletic Manager, Sophomore Representative, Off-Campus Representative, two BoC representatives, CRC representative, ARC representative. The election for each office shall be held before nominations for successive offices are closed. Election shall require a majority of votes cast from the total voting membership, not including blank ballots, and they shall be made by secret ballot. Elections shall be managed by an Election Committee in accordance with the Procedural Amendment. \\
	3. Election for Social Team and Athletic Team shall be carried out through proxy voting beginning at the conclusion of the House Meeting described in section 2.4.2 and concluding 48 hours later. The elections shall be managed by the Election Committee. 
	\item \textbf{Proposed Language:} \\
	2.4 Nominations, Elections, and Appointments \\
	2. Election of all officers except Social Team and Athletic Team shall be carried out at a House Meeting held for that purpose within one week after the nomination meeting. The order of election shall be in the following order: President, Vice President, Secretary, Treasurer, Librarian, Social Manager, Athletic Manager, Sophomore Representative, Off-Campus Representative, two BoC representatives, CRC representative, ARC representative. \textcolor{red}{The election for each office shall be held before nominations for successive offices are closed. Election shall require a majority of votes cast from the total voting membership, not including blank ballots, and they shall be made by secret ballot.} Elections shall be managed by an Election Committee in accordance with the Procedural Amendment. \\
	3. Election for Social Team and Athletic Team shall be carried out through proxy voting beginning \textcolor{red}{at the conclusion of the 36-hour grievance period in accordance with the Procedural Amendment and concluding a week} later. The elections shall be managed by the Election Committee. 
\end{itemize}

\section{Records Amendment}
\begin{itemize}
	\item \textbf{Date Passed:} 5/26/2020
	\item \textbf{Author:} Elaine Lowinger
	\item \textbf{Reasoning:} There is no current record of the amendments proposed that have modified the constitution. Therefore, many times when proposing new amendments or trying to determine intent of the constitution, knowledge of this document and the phrasing is lost in time. The goal of this is to log the change in the constitution in an accessible form to the house to understand the changes made. 
	\item \textbf{Current Language:} None
	\item \textbf{Proposed Language:} \\
	\textcolor{ForestGreen}{
	8.4 Records \\
	If a proposed amendment is passed, the Constitution is modified to match the amended language. In addition, the proposal for the amendment shall be added to the “Amendment History” document. This includes the date of the amendment's approval, the proposer of the amendment, the reasoning of the amendment, the unamended language, and the approved amended language. The Amendment History document shall be accessible to all Full Members of the House.}
\end{itemize}

\section{IMSS Representative Amendment}

\begin{itemize}
	\item \textbf{Date Passed:} 4/11/2020
	\item \textbf{Author:} Chase Blanchette
	\item \textbf{Reasoning:} The duties of the House IMSS reps are currently not codified in the constitution, and no roles for upkeep of specific House technology (3D printer(s), printers) are assigned, which could lead to a lack of accountability. This amendment outline the responsibilities of the IMSS reps and proposes individual roles among the IMSS reps to ensure accountability. 
	\item \textbf{Current Language:} None
	\item \textbf{Proposed Language:} \\
	2.3 Duties and Qualifications of Appointed Officers \\
	\textcolor{ForestGreen}{
	7. The House IMSS Representatives shall act as liaisons between House members and IMSS for network, application, hardware, and software problems. The IMSS Representatives shall include the following: 
	\begin{enumerate}[label=(\alph*)]
		\item The Lab Rat shall be responsible for the upkeep of computer lab computers, organization of computer lab supplies, and cleaning of the space.
		\item The Printer Representative shall ensure that the House paper-and-ink printers are in working order for use by House members and coordinate with admin for replacement parts and supplies.
		\item The 3D Printer Representative shall ensure that the House 3D printers are in working order for use by House members, procure House-provided 3D printer supplies, and maintain a usage policy for the printer/supplies.
		\item The Webmaster shall be responsible for administration of the Ruddock web server and keeping the website (ruddock.caltech.edu) online.
	\end{enumerate}
	}
	
\end{itemize}




\section{Abstain Amendment}
\begin{itemize}
	\item \textbf{Date Passed:} 5/21/2019
	\item \textbf{Author:} Elaine Lowinger and Sierra Lopezalles
	\item \textbf{Reasoning:} The current issues with House Memberships, as explained by the UCCs reviewing the constitution, is that there is no distinction between voting No and abstaining from voting, as the wording says “only 2/3 majority of those voting”. To allow a voice to people in the house who would like to be present at meetings and vote, but not comfortable making a firm decision, we want to give them the option to abstain. To do so, we have changed the wording of the constitution. 
	\item \textbf{Current Language:} \\
	1.2 Membership \\
	1. The types and qualifications of membership are: \\
	(a) a Full Member, who is an undergraduate of the California Institute of Technology and is picked as such at the end of Rotation Week, or is elected as such by a 2/3 majority of those voting at a House meeting. 
	\item \textbf{Proposed Language:} \\
	1.2 Membership \\
	1. The types and qualifications of membership are: \\
	(a) a Full Member, who is an undergraduate of the California Institute of Technology and is picked as such at the end of Rotation Week, or is elected at a House meeting by a 2/3 majority of \textcolor{ForestGreen}{non-abstain votes with at least 50\% of the members present voting “Yes.”}
\end{itemize}



\section{Room Occupancy Priority Amendment}
\begin{itemize}
	\item \textbf{Date Passed:} 3/19/2019
	\item \textbf{Author:} Emily Wu
	\item \textbf{Reasoning:} Room Hassle and the time leading up to it can be very stressful for Rudds who want to live in the house, especially for those without a position that has a room pick. In order to maximize the number of members that can live in the house and reduce this stress, I am making it explicit that Rudds opting for singles get least priority during Room Hassle, and clarifying some of the other language regarding room occupancy priority.  
	\item \textbf{Current Language:} \\
	7.1.4. Rooms, with the exception of a designated sophomore off-campus alley, shall be picked in the following order: 
	\begin{enumerate}[label=(\alph*)]
		\item by the House President, then 
		\item by all other Full Members in order of 
		\begin{enumerate}[label=(\roman*)]
			\item class (seniors then juniors...), then 
			\item intended number of occupants as a percentage of the room capacity, descending, then 
			\item office as specified in section 7.2.8, then 
			\item up to two random sophomores as determined by a random order prepared prior to the section 7.1.2a Room Hassle in a manner prescribed by the Secretary, then 
			\item a random order prepared prior to the section 7.1.2a Room Hassle in a manner prescribed by the Secretary.
		\end{enumerate}
	\end{enumerate}
	\item \textbf{Proposed Language:} \\
	7.1.4. Rooms, with the exception of a designated sophomore off-campus alley, shall be picked in the following order: 
	\begin{enumerate}[label=(\alph*)]
		\item by the House President, then 
		\item by all other Full Members in order of 
		\begin{enumerate}[label=(\roman*)]
			\item class (seniors then juniors...) \textcolor{ForestGreen}{with the exception of partial occupancy of room capacity}, then 
			\item intended number of occupants \textcolor{ForestGreen}{in excess of room capacity} as a percentage of the room capacity, descending, then 
			\item office as specified in section 7.2.8, then 
			\item up to two random sophomores as determined by a random order prepared prior to the section 7.1.2a Room Hassle in a manner prescribed by the Secretary, then 
			\item a random order prepared prior to the section 7.1.2a Room Hassle in a manner prescribed by the Secretary\textcolor{ForestGreen}{, then}
			\item \textcolor{ForestGreen}{partial occupancy of room capacity.}
		\end{enumerate}
	\end{enumerate}
\end{itemize}

\section{Bechtel Freshmen Room Hassle Priority Amendment}
\begin{itemize}
	\item \textbf{Date Passed:} 3/19/2019
	\item \textbf{Author:} Emily Wu
	\item \textbf{Reasoning:} This amendment is intended to address the case of freshmen who lived in Bechtel their first year, but would like to have the opportunity to live in the house their sophomore year in years where Room Hassle is competitive. Since 30/34 of Ruddock freshmen (as of this year) have the opportunity to live in a House their freshman year, this amendment seeks to give freshmen who have not yet had this experience an opportunity to. 
	\item \textbf{Current Language:} \\
	7.1.4. Rooms, with the exception of a designated sophomore off-campus alley, shall be picked in the following order: 
	\begin{enumerate}[label=(\alph*)]
		\item by the House President, then 
		\item by all other Full Members in order of 
		\begin{enumerate}[label=(\roman*)]
			\item class (seniors then juniors...), then 
			\item intended number of occupants as a percentage of the room capacity, descending, then 
			\item office as specified in section 7.2.8, then 
			\item up to two random sophomores as determined by a random order prepared prior to the section 7.1.2a Room Hassle in a manner prescribed by the Secretary, then 
			\item a random order prepared prior to the section 7.1.2a Room Hassle in a manner prescribed by the Secretary.
		\end{enumerate}
	\end{enumerate}
	\item \textbf{Proposed Language:} \\
	7.1.4. Rooms, with the exception of a designated sophomore off-campus alley, shall be picked in the following order: 
	\begin{enumerate}[label=(\alph*)]
		\item by the House President, then 
		\item by all other Full Members in order of 
		\begin{enumerate}[label=(\roman*)]
			\item class (seniors then juniors...), then 
			\item intended number of occupants as a percentage of the room capacity, descending, then 
			\item office as specified in section 7.2.8, then 
			\item up to two random sophomores as determined by a random order prepared prior to the section 7.1.2a Room Hassle in a manner prescribed by the Secretary\textcolor{ForestGreen}{. Rising sophomores who have not lived in a House are given priority. T}hen 
			\item a random order prepared prior to the section 7.1.2a Room Hassle in a manner prescribed by the Secretary.
		\end{enumerate}
	\end{enumerate}
\end{itemize}

\section{Room Hassle ROCA Removal Amendment}
\begin{itemize}
	\item \textbf{Date Passed:} 3/19/2019
	\item \textbf{Author:} Emily Wu
	\item \textbf{Reasoning:} We no longer have a house designated as our ROCA (Ruddock Off Campus Alley), therefore we no longer need the language describing its picks procedure.  
	\item \textbf{Current Language:} \\
	7.1.4. Rooms, with the exception of a designated sophomore off-campus alley, shall be picked in the following order: 
	\begin{enumerate}[label=(\alph*)]
		\item by the House President, then 
		\item by all other Full Members in order of 
		\begin{enumerate}[label=(\roman*)]
			\item class (seniors then juniors...), then 
			\item intended number of occupants as a percentage of the room capacity, descending, then 
			\item office as specified in section 7.2.8, then 
			\item up to two random sophomores as determined by a random order prepared prior to the section 7.1.2a Room Hassle in a manner prescribed by the Secretary, then 
			\item a random order prepared prior to the section 7.1.2a Room Hassle in a manner prescribed by the Secretary.
		\end{enumerate}
	\end{enumerate}
	
	7.15. Prior to the section 7.1.2a Room Hassle the Secretary shall designate an off-campus alley for the members of the sophomore class. The residency shall be filled in the following manner: 
	\begin{enumerate}[label=(\alph*)]
		\item intended number of occupants, then 
		\item intended number of sophomore occupants, then 
		\item sophomores' office as specified in section 7.2.8, then 
		\item sophomores' random order prepared prior to the section 7.1.2a Room Hassle in a manner proscribed by the Secretary. 
	\end{enumerate}
	
	7.1.6. In the event that no members of the sophomore class wish to reside within the off-campus alley, the standard pick order shall be used.
	
	\item \textbf{Proposed Language:} \\
	7.1.4. Rooms\textcolor{red}{, with the exception of a designated sophomore off-campus alley,} shall be picked in the following order: 
	\begin{enumerate}[label=(\alph*)]
		\item by the House President, then 
		\item by all other Full Members in order of 
		\begin{enumerate}[label=(\roman*)]
			\item class (seniors then juniors...), then 
			\item intended number of occupants as a percentage of the room capacity, descending, then 
			\item office as specified in section 7.2.8, then 
			\item up to two random sophomores as determined by a random order prepared prior to the section 7.1.2a Room Hassle in a manner prescribed by the Secretary, then 
			\item a random order prepared prior to the section 7.1.2a Room Hassle in a manner prescribed by the Secretary.
		\end{enumerate}
	\end{enumerate}
	\textcolor{red}{
		7.15. Prior to the section 7.1.2a Room Hassle the Secretary shall designate an off-campus alley for the members of the sophomore class. The residency shall be filled in the following manner: 
		\begin{enumerate}[label=(\alph*)]
			\item intended number of occupants, then 
			\item intended number of sophomore occupants, then 
			\item sophomores' office as specified in section 7.2.8, then 
			\item sophomores' random order prepared prior to the section 7.1.2a Room Hassle in a manner proscribed by the Secretary. 
		\end{enumerate}
	}
	\textcolor{red}{
		7.1.6. In the event that no members of the sophomore class wish to reside within the off-campus alley, the standard pick order shall be used.
	}
\end{itemize}

\section{Diversity Council Amendment}
\begin{itemize}
	\item \textbf{Date Passed:} 2/17/2019
	\item \textbf{Author:} Sarah Jeoung
	\item \textbf{Reasoning:} The proposed establishment of the Diversity Council is intended to help with coverage in Ruddock's support system. The Diversity Council shall help in ensuring representation of underrepresented minorities both during the selection of the UCC team and in general. A past member of the Diversity Council will help select the new Council. 
	\item \textbf{Current Language:} \\
	2.3 Duties and Qualifications of Appointed Officers \\
	None \\
	2.4 Nominations, Elections, and Appointments \\
	5. Upper Class Committeeman sign-ups shall be posted concurrently with Peer Advocate signups, and all Upper Class Committeemen and Peer Advocates shall be appointed at the same meeting. The entire Executive Committee and the current Head Upper Class Committeeman will be present for selection, and Peer Advocates selected will be members of the Upper Class Committeeman team. The term of service for those selected to be members of the Upper Class Committeeman team will be academic year directly following their appointment, with the exception of the newly-appointed Head Upper Class Committeeman, who will begin serving in their role during Third Term. 
	\item \textbf{Proposed Language:} \\
	2.3 Duties and Qualifications of Appointed Officers \\
	This part of the amendment will go between the current sections 2.3.2 and 2.3.3, becoming the new 2.3.3. \\
	\textcolor{ForestGreen}{3. The Diversity Councilmembers shall be responsible for liaising with the Caltech Center for Diversity. They shall be available to advise any member of the House on issues pertaining to diversity. They shall select one Representative to be present during the selection of the Upper Class Committee and the Diversity Council. This Representative may not reapply for a position on the succeeding Diversity Council. In the case where no suitable volunteers can be found, the Head Upper Class Committee-person shall act as a proxy for the Diversity Council during the selection of the Upper Class Committee and the Diversity Council.} \\
	2.4 Nominations, Elections, and Appointments \\
	5. Upper Class Committee\textcolor{ForestGreen}{-member} sign-ups\textcolor{ForestGreen}{,} \textcolor{red}{shall be posted concurrently with} Peer Advocate signups, and \textcolor{ForestGreen}{Diversity Councilmember sign-ups shall be posted concurrently. A}ll Upper Class Committee\textcolor{ForestGreen}{-members, } Peer Advocates\textcolor{ForestGreen}{, and Diversity Councilmembers} shall be appointed at the same meeting. The entire Executive Committee\textcolor{ForestGreen}{,} the current Head Upper Class Committee\textcolor{ForestGreen}{-member, and the Diversity Council Representative} will be present for selection\textcolor{ForestGreen}{. Selected} Peer Advocates will be members of the Upper Class Committee\textcolor{red}{man team}. The term of service for those selected to be members of the Upper Class Committee\textcolor{red}{man team} will be \textcolor{ForestGreen}{the} academic year directly following their appointment, with the exception of the newly-appointed Head Upper Class Committee\textcolor{ForestGreen}{-member and Diversity Council-members}, who will begin serving in their role\textcolor{ForestGreen}{s} during Third Term.
\end{itemize}

\section{Gender Neutral Wording}
\begin{itemize}
	\item \textbf{Date Passed:} 1/19/2019
	\item \textbf{Author:} Emily Wu and Erik Herrera
	\item \textbf{Reasoning:} The current version of the constitution uses only ‘He/him/his' pronouns. We would like to use more inclusive language. This would benefit current Ruddock House members and all the classes to come. 
	\item \textbf{Current Language:} Numerous instances. The change will apply to the entire constitution.
	\item \textbf{Proposed Language:} Change everywhere the constitution uses male-centered language to gender-neutral language. Examples include \textcolor{red}{he/him/his} to genderless pronouns (\textcolor{ForestGreen}{they/them}) or avoid pronoun use, Committee\textcolor{red}{man} to Committee\textcolor{ForestGreen}{-member}, and fresh\textcolor{red}{man} to \textcolor{ForestGreen}{frosh}. 
\end{itemize}

\section{Splitting Steward Role Between Veep and Treasurer}
\begin{itemize}
	\item \textbf{Date Passed:} 1/19/2019
	\item \textbf{Author:} Rupesh Jeyaram
	\item \textbf{Reasoning:}  As of now, Veep can assume more House responsibilities. The lack of Veep duties results from the Veep/HUCC split two years ago. Treasurer already has a lot of responsibilities, and it would be nice to distribute work more evenly. Makes sense to shift responsibilities in this manner.
	\item \textbf{Current Language:} \\
	2.2 Qualifications and Duties of Elected Officers \\
	2. The Vice President shall be a junior or senior for the latter portion of his term in office. He shall assume the duties of the President in the event of the President's absence. He shall be responsible for maintaining the external relations of the House, for bringing guests to the House, and shall act as official host for all House guests. He shall be the Ruddock representative on the Review Committee. \\
	4. The Treasurer shall be responsible to the House for all receipts and expenditures, shall maintain an adequate book-keeping system, shall submit a report to the Executive Committee at each regularly scheduled meeting, shall propose a budget to the Executive Committee for each term, and shall submit to the House a written report of the finances at the end of each term. 
	\item \textbf{Proposed Language:} \\
	2.2 Qualifications and Duties of Elected Officers \\
	2. The Vice President shall be a junior or senior for the latter portion of his term in office. He shall assume the duties of the President in the event of the President's absence. He shall be responsible for maintaining the external relations of the House, for bringing guests to the House, and shall act as official host for all House guests. He shall be the Ruddock representative on the Review Committee. \textcolor{ForestGreen}{They shall share duties with the Treasurer in maintaining house property.} \\
	4. The Treasurer shall be responsible to the House for all receipts and expenditures, shall maintain an adequate book-keeping system, shall submit a report to the Executive Committee at each regularly scheduled meeting, shall propose a budget to the Executive Committee for each term, and shall submit to the House a written report of the finances at the end of each term. \textcolor{ForestGreen}{They shall share duties with the Vice President in maintaining house property.}
\end{itemize}

\end{document}

