\documentclass[10pt]{article} % use larger type; default would be 10pt
\usepackage[utf8]{inputenc} % set input encoding (not needed with XeLaTeX)

\usepackage{geometry} % to change the page dimensions
\geometry{letterpaper}
\usepackage{graphicx} % support the \includegraphics command and options
\setlength\voffset{-0.2in}

\usepackage[dvipsnames]{xcolor} % allows font color change
\usepackage{enumitem} % allows changing enumeration for sublists to match constitution numbering

\title{\includegraphics[width=1.55in]{ruddock_crest.png} \\ Constitutional Amendment History}
\author{Ruddock House}
\date{\today} % Displays date of latest build


%%%%%%%%%% Why this document exists %%%%%%%%%%
% The Amendment History document was created as a result of passage of a constitutional amendment written by Elaine Lowinger to keep track of changes to the constitution and the reasoning behind them: 
%    8.4 Records
%    If a proposed amendment is passed, the Constitution is modified to match the amended language. In addition, the proposal for the amendment shall be added to the “Amendment History” document. This includes the date of the amendment’s approval, the proposer of the amendment, the reasoning of the amendment, the unamended language, and the approved amended language. The Amendment History document shall be accessible to all Full Members of the House.
% As a result, this document started with four amendments passed on May 26, 2020 (including the Records Amendment) and retroactively includes amendments passed by January 19, 2019. Amendments are in reverse chronological order (newest to oldest).
% It is the duty of the House Secretary to update the constitution and this Amendment History document.


%%%%%%%%%% Structure of Each Entry %%%%%%%%%%
% Each new amendment entry must include:
% 1) the date of the amendment’s approval, 
% 2) the proposer of the amendment, 
% 3) the reasoning of the amendment, 
% 4) the existing language, and 
% 5) the approved amended language.
% Note: Use \\ or \newline for line breaks in each bullet point.
%
% Sample Entry:
% \section{Amendment Name}
% \begin{itemize}
% 	\item \textbf{Date Passed:} date
% 	\item \textbf{Author:} name
% 	\item \textbf{Reasoning:} explanation 
% 	\item \textbf{Current Language:} What is currently written in the constitution. Use section numbers and the text, as numbering will shift with further amendments to the constitution.
% 	\item \textbf{Proposed Language:} Changes made from current language, with additions in \textcolor{ForestGreen}{green} and subtractions in \textcolor{red}{red}.
% \end{itemize}


\begin{document}

\maketitle

\section{Food Representative Amendment}
\begin{itemize}
	\item \textbf{Date Passed:} 5/26/2020
	\item \textbf{Author:} Sierra Lopezalles
	\item \textbf{Reasoning:} The position of Food Representative is not currently in the constitution, this amendment both adds Food Rep to the constitution and makes it a part of the role of being Head Waiter. The Head Waiter is the person best suited to being Food Rep since they are present at most dinners and are best able to judge the house’s response to meals.
	\item \textbf{Current Language:} \\
	2.3 Duties and Qualifications of Appointed Officers \\
	5. The Head Waiter shall be responsible for maintaining proper decorum and conduct in the dining room, and shall direct the student waiters.
	\item \textbf{Proposed Language:} \\
	2.3 Duties and Qualifications of Appointed Officers \\
	5. The Head Waiter shall be responsible for maintaining proper decorum and conduct in the dining room, and shall direct the student waiters. \textcolor{ForestGreen}{The Head Waiter shall be the Food Representative for Ruddock House.}
\end{itemize}

\section{Election Amendment for Online Voting}
\begin{itemize}
	\item \textbf{Date Passed:} 5/26/2020
	\item \textbf{Author:} Sierra Lopezalles
	\item \textbf{Reasoning:} Currently Electcomm uses an online google form for proxy voting. This amendment would align the constitution to the procedure that Electcomm is already using.
	\item \textbf{Current Language:} \\
	A Procedural Amendment \\
	Proxy votes shall be submitted to a member of the Election Committee in written form prior to the election. The Election Committee shall be responsible for entering proxy votes during the election.
	\item \textbf{Proposed Language:} \\
	A Procedural Amendment \\
	Proxy votes shall be submitted to a member of the Election Committee \textcolor{red}{in written form} prior to the election. The Election Committee shall be responsible for entering proxy votes during the election. 
\end{itemize}

\section{Election Amendment for Timeline Modifications}
\begin{itemize}
	\item \textbf{Date Passed:} 5/26/2020
	\item \textbf{Author:} Sierra Lopezalles
	\item \textbf{Reasoning:} \\
	This amendment fixes outdated language regarding the timing of elections. In the past, each position was nominated and elected before the next position could begin. Since we have moved to electing all positions at the same meeting, this text must also be removed in order to align the constitution to our current procedures. \\
	This amendment also adjusts the timeline for electing Soc Team and Ath Team. This ensures that a Soc Man and Ath Man will have been elected before voting on the team commences. Additionally, it extends the voting period to a week, as has been done in the past 2 years. \\
	This amendment also removes text about requiring a majority of votes since this is conflict with the Procedural Amendment. \\
	The Procedural Amendment also requires that ballots be submitted anonymously, thus it does change the procedure to remove the part about secret ballots.
	\item \textbf{Current Language:} \\
	2.3. Duties and Qualifications of Appointed Officers \\
	5. The Head Waiter shall be responsible for maintaining proper decorum and conduct in the dining room, and shall direct the student waiters.2.4 Nominations, Elections, and Appointments \\
	2. Election of all officers except Social Team and Athletic Team shall be carried out at a House Meeting held for that purpose within one week after the nomination meeting. The order of election shall be in the following order: President, Vice President, Secretary, Treasurer, Librarian, Social Manager, Athletic Manager, Sophomore Representative, Off-Campus Representative, two BoC representatives, CRC representative, ARC representative. The election for each office shall be held before nominations for successive offices are closed. Election shall require a majority of votes cast from the total voting membership, not including blank ballots, and they shall be made by secret ballot. Elections shall be managed by an Election Committee in accordance with the Procedural Amendment. \\
	3. Election for Social Team and Athletic Team shall be carried out through proxy voting beginning at the conclusion of the House Meeting described in section 2.4.2 and concluding 48 hours later. The elections shall be managed by the Election Committee. 
	\item \textbf{Proposed Language:} \\
	2.4 Nominations, Elections, and Appointments \\
	2. Election of all officers except Social Team and Athletic Team shall be carried out at a House Meeting held for that purpose within one week after the nomination meeting. The order of election shall be in the following order: President, Vice President, Secretary, Treasurer, Librarian, Social Manager, Athletic Manager, Sophomore Representative, Off-Campus Representative, two BoC representatives, CRC representative, ARC representative. \textcolor{red}{The election for each office shall be held before nominations for successive offices are closed. Election shall require a majority of votes cast from the total voting membership, not including blank ballots, and they shall be made by secret ballot.} Elections shall be managed by an Election Committee in accordance with the Procedural Amendment. \\
	3. Election for Social Team and Athletic Team shall be carried out through proxy voting beginning \textcolor{red}{at the conclusion of the 36-hour grievance period in accordance with the Procedural Amendment and concluding a week} later. The elections shall be managed by the Election Committee. 
\end{itemize}

\section{Records Amendment}
\begin{itemize}
	\item \textbf{Date Passed:} 5/26/2020
	\item \textbf{Author:} Elaine Lowinger
	\item \textbf{Reasoning:} There is no current record of the amendments proposed that have modified the constitution. Therefore, many times when proposing new amendments or trying to determine intent of the constitution, knowledge of this document and the phrasing is lost in time. The goal of this is to log the change in the constitution in an accessible form to the house to understand the changes made. 
	\item \textbf{Current Language:} None
	\item \textbf{Proposed Language:} \\
	\textcolor{ForestGreen}{
	8.4 Records \\
	If a proposed amendment is passed, the Constitution is modified to match the amended language. In addition, the proposal for the amendment shall be added to the “Amendment History” document. This includes the date of the amendment’s approval, the proposer of the amendment, the reasoning of the amendment, the unamended language, and the approved amended language. The Amendment History document shall be accessible to all Full Members of the House.}
\end{itemize}

\section{IMSS Representative Amendment}

\begin{itemize}
	\item \textbf{Date Passed:} 4/11/2020
	\item \textbf{Author:} Chase Blanchette
	\item \textbf{Reasoning:} The duties of the House IMSS reps are currently not codified in the constitution, and no roles for upkeep of specific House technology (3D printer(s), printers) are assigned, which could lead to a lack of accountability. This amendment outline the responsibilities of the IMSS reps and proposes individual roles among the IMSS reps to ensure accountability. 
	\item \textbf{Current Language:} None
	\item \textbf{Proposed Language:} \\
	2.3 Duties and Qualifications of Appointed Officers \\
	\textcolor{ForestGreen}{
	7. The House IMSS Representatives shall act as liaisons between House members and IMSS for network, application, hardware, and software problems. The IMSS Representatives shall include the following: 
	\begin{enumerate}[label=(\alph*)]
		\item The Lab Rat shall be responsible for the upkeep of computer lab computers, organization of computer lab supplies, and cleaning of the space.
		\item The Printer Representative shall ensure that the House paper-and-ink printers are in working order for use by House members and coordinate with admin for replacement parts and supplies.
		\item The 3D Printer Representative shall ensure that the House 3D printers are in working order for use by House members, procure House-provided 3D printer supplies, and maintain a usage policy for the printer/supplies.
		\item The Webmaster shall be responsible for administration of the Ruddock web server and keeping the website (ruddock.caltech.edu) online.
	\end{enumerate}
	}
	
\end{itemize}




\section{Abstain Amendment}
\begin{itemize}
	\item \textbf{Date Passed:} 5/21/2019
	\item \textbf{Author:} Elaine Lowinger and Sierra Lopezalles
	\item \textbf{Reasoning:} The current issues with House Memberships, as explained by the UCCs reviewing the constitution, is that there is no distinction between voting No and abstaining from voting, as the wording says “only 2/3 majority of those voting”. To allow a voice to people in the house who would like to be present at meetings and vote, but not comfortable making a firm decision, we want to give them the option to abstain. To do so, we have changed the wording of the constitution. 
	\item \textbf{Current Language:} \\
	1.2 Membership \\
	1. The types and qualifications of membership are: \\
	(a) a Full Member, who is an undergraduate of the California Institute of Technology and is picked as such at the end of Rotation Week, or is elected as such by a 2/3 majority of those voting at a House meeting. 
	\item \textbf{Proposed Language:} \\
	1.2 Membership \\
	1. The types and qualifications of membership are: \\
	(a) a Full Member, who is an undergraduate of the California Institute of Technology and is picked as such at the end of Rotation Week, or is elected at a House meeting by a 2/3 majority of \textcolor{ForestGreen}{non-abstain votes with at least 50\% of the members present voting “Yes.”}
\end{itemize}



\section{Room Occupancy Priority Amendment}
\begin{itemize}
	\item \textbf{Date Passed:} 3/19/2019
	\item \textbf{Author:} Emily Wu
	\item \textbf{Reasoning:} Room Hassle and the time leading up to it can be very stressful for Rudds who want to live in the house, especially for those without a position that has a room pick. In order to maximize the number of members that can live in the house and reduce this stress, I am making it explicit that Rudds opting for singles get least priority during Room Hassle, and clarifying some of the other language regarding room occupancy priority.  
	\item \textbf{Current Language:} \\
	7.1.4. Rooms, with the exception of a designated sophomore off-campus alley, shall be picked in the following order: 
	\begin{enumerate}[label=(\alph*)]
		\item by the House President, then 
		\item by all other Full Members in order of 
		\begin{enumerate}[label=(\roman*)]
			\item class (seniors then juniors...), then 
			\item intended number of occupants as a percentage of the room capacity, descending, then 
			\item office as specified in section 7.2.8, then 
			\item up to two random sophomores as determined by a random order prepared prior to the section 7.1.2a Room Hassle in a manner prescribed by the Secretary, then 
			\item a random order prepared prior to the section 7.1.2a Room Hassle in a manner prescribed by the Secretary.
		\end{enumerate}
	\end{enumerate}
	\item \textbf{Proposed Language:} \\
	7.1.4. Rooms, with the exception of a designated sophomore off-campus alley, shall be picked in the following order: 
	\begin{enumerate}[label=(\alph*)]
		\item by the House President, then 
		\item by all other Full Members in order of 
		\begin{enumerate}[label=(\roman*)]
			\item class (seniors then juniors...) \textcolor{ForestGreen}{with the exception of partial occupancy of room capacity}, then 
			\item intended number of occupants \textcolor{ForestGreen}{in excess of room capacity} as a percentage of the room capacity, descending, then 
			\item office as specified in section 7.2.8, then 
			\item up to two random sophomores as determined by a random order prepared prior to the section 7.1.2a Room Hassle in a manner prescribed by the Secretary, then 
			\item a random order prepared prior to the section 7.1.2a Room Hassle in a manner prescribed by the Secretary\textcolor{ForestGreen}{, then}
			\item \textcolor{ForestGreen}{partial occupancy of room capacity.}
		\end{enumerate}
	\end{enumerate}
\end{itemize}

\section{Bechtel Freshmen Room Hassle Priority Amendment}
\begin{itemize}
	\item \textbf{Date Passed:} 3/19/2019
	\item \textbf{Author:} Emily Wu
	\item \textbf{Reasoning:} This amendment is intended to address the case of freshmen who lived in Bechtel their first year, but would like to have the opportunity to live in the house their sophomore year in years where Room Hassle is competitive. Since 30/34 of Ruddock freshmen (as of this year) have the opportunity to live in a House their freshman year, this amendment seeks to give freshmen who have not yet had this experience an opportunity to. 
	\item \textbf{Current Language:} \\
	7.1.4. Rooms, with the exception of a designated sophomore off-campus alley, shall be picked in the following order: 
	\begin{enumerate}[label=(\alph*)]
		\item by the House President, then 
		\item by all other Full Members in order of 
		\begin{enumerate}[label=(\roman*)]
			\item class (seniors then juniors...), then 
			\item intended number of occupants as a percentage of the room capacity, descending, then 
			\item office as specified in section 7.2.8, then 
			\item up to two random sophomores as determined by a random order prepared prior to the section 7.1.2a Room Hassle in a manner prescribed by the Secretary, then 
			\item a random order prepared prior to the section 7.1.2a Room Hassle in a manner prescribed by the Secretary.
		\end{enumerate}
	\end{enumerate}
	\item \textbf{Proposed Language:} \\
	7.1.4. Rooms, with the exception of a designated sophomore off-campus alley, shall be picked in the following order: 
	\begin{enumerate}[label=(\alph*)]
		\item by the House President, then 
		\item by all other Full Members in order of 
		\begin{enumerate}[label=(\roman*)]
			\item class (seniors then juniors...), then 
			\item intended number of occupants as a percentage of the room capacity, descending, then 
			\item office as specified in section 7.2.8, then 
			\item up to two random sophomores as determined by a random order prepared prior to the section 7.1.2a Room Hassle in a manner prescribed by the Secretary\textcolor{ForestGreen}{. Rising sophomores who have not lived in a House are given priority. T}hen 
			\item a random order prepared prior to the section 7.1.2a Room Hassle in a manner prescribed by the Secretary.
		\end{enumerate}
	\end{enumerate}
\end{itemize}

\section{Room Hassle ROCA Removal Amendment}
\begin{itemize}
	\item \textbf{Date Passed:} 3/19/2019
	\item \textbf{Author:} Emily Wu
	\item \textbf{Reasoning:} We no longer have a house designated as our ROCA (Ruddock Off Campus Alley), therefore we no longer need the language describing its picks procedure.  
	\item \textbf{Current Language:} \\
	7.1.4. Rooms, with the exception of a designated sophomore off-campus alley, shall be picked in the following order: 
	\begin{enumerate}[label=(\alph*)]
		\item by the House President, then 
		\item by all other Full Members in order of 
		\begin{enumerate}[label=(\roman*)]
			\item class (seniors then juniors...), then 
			\item intended number of occupants as a percentage of the room capacity, descending, then 
			\item office as specified in section 7.2.8, then 
			\item up to two random sophomores as determined by a random order prepared prior to the section 7.1.2a Room Hassle in a manner prescribed by the Secretary, then 
			\item a random order prepared prior to the section 7.1.2a Room Hassle in a manner prescribed by the Secretary.
		\end{enumerate}
	\end{enumerate}
	
	7.15. Prior to the section 7.1.2a Room Hassle the Secretary shall designate an off-campus alley for the members of the sophomore class. The residency shall be filled in the following manner: 
	\begin{enumerate}[label=(\alph*)]
		\item intended number of occupants, then 
		\item intended number of sophomore occupants, then 
		\item sophomores’ office as specified in section 7.2.8, then 
		\item sophomores’ random order prepared prior to the section 7.1.2a Room Hassle in a manner proscribed by the Secretary. 
	\end{enumerate}
	
	7.1.6. In the event that no members of the sophomore class wish to reside within the off-campus alley, the standard pick order shall be used.
	
	\item \textbf{Proposed Language:} \\
	7.1.4. Rooms\textcolor{red}{, with the exception of a designated sophomore off-campus alley,} shall be picked in the following order: 
	\begin{enumerate}[label=(\alph*)]
		\item by the House President, then 
		\item by all other Full Members in order of 
		\begin{enumerate}[label=(\roman*)]
			\item class (seniors then juniors...), then 
			\item intended number of occupants as a percentage of the room capacity, descending, then 
			\item office as specified in section 7.2.8, then 
			\item up to two random sophomores as determined by a random order prepared prior to the section 7.1.2a Room Hassle in a manner prescribed by the Secretary, then 
			\item a random order prepared prior to the section 7.1.2a Room Hassle in a manner prescribed by the Secretary.
		\end{enumerate}
	\end{enumerate}
	\textcolor{red}{
		7.15. Prior to the section 7.1.2a Room Hassle the Secretary shall designate an off-campus alley for the members of the sophomore class. The residency shall be filled in the following manner: 
		\begin{enumerate}[label=(\alph*)]
			\item intended number of occupants, then 
			\item intended number of sophomore occupants, then 
			\item sophomores’ office as specified in section 7.2.8, then 
			\item sophomores’ random order prepared prior to the section 7.1.2a Room Hassle in a manner proscribed by the Secretary. 
		\end{enumerate}
	}
	\textcolor{red}{
		7.1.6. In the event that no members of the sophomore class wish to reside within the off-campus alley, the standard pick order shall be used.
	}
\end{itemize}

\section{Diversity Council Amendment}
\begin{itemize}
	\item \textbf{Date Passed:} 2/17/2019
	\item \textbf{Author:} Sarah Jeoung
	\item \textbf{Reasoning:} The proposed establishment of the Diversity Council is intended to help with coverage in Ruddock’s support system. The Diversity Council shall help in ensuring representation of underrepresented minorities both during the selection of the UCC team and in general. A past member of the Diversity Council will help select the new Council. 
	\item \textbf{Current Language:} \\
	2.3 Duties and Qualifications of Appointed Officers \\
	None \\
	2.4 Nominations, Elections, and Appointments \\
	5. Upper Class Committeeman sign-ups shall be posted concurrently with Peer Advocate signups, and all Upper Class Committeemen and Peer Advocates shall be appointed at the same meeting. The entire Executive Committee and the current Head Upper Class Committeeman will be present for selection, and Peer Advocates selected will be members of the Upper Class Committeeman team. The term of service for those selected to be members of the Upper Class Committeeman team will be academic year directly following their appointment, with the exception of the newly-appointed Head Upper Class Committeeman, who will begin serving in their role during Third Term. 
	\item \textbf{Proposed Language:} \\
	2.3 Duties and Qualifications of Appointed Officers \\
	This part of the amendment will go between the current sections 2.3.2 and 2.3.3, becoming the new 2.3.3. \\
	\textcolor{ForestGreen}{3. The Diversity Councilmembers shall be responsible for liaising with the Caltech Center for Diversity. They shall be available to advise any member of the House on issues pertaining to diversity. They shall select one Representative to be present during the selection of the Upper Class Committee and the Diversity Council. This Representative may not reapply for a position on the succeeding Diversity Council. In the case where no suitable volunteers can be found, the Head Upper Class Committee-person shall act as a proxy for the Diversity Council during the selection of the Upper Class Committee and the Diversity Council.} \\
	2.4 Nominations, Elections, and Appointments \\
	5. Upper Class Committee\textcolor{ForestGreen}{-member} sign-ups\textcolor{ForestGreen}{,} \textcolor{red}{shall be posted concurrently with} Peer Advocate signups, and \textcolor{ForestGreen}{Diversity Councilmember sign-ups shall be posted concurrently. A}ll Upper Class Committee\textcolor{ForestGreen}{-members, } Peer Advocates\textcolor{ForestGreen}{, and Diversity Councilmembers} shall be appointed at the same meeting. The entire Executive Committee\textcolor{ForestGreen}{,} the current Head Upper Class Committee\textcolor{ForestGreen}{-member, and the Diversity Council Representative} will be present for selection\textcolor{ForestGreen}{. Selected} Peer Advocates will be members of the Upper Class Committee\textcolor{red}{man team}. The term of service for those selected to be members of the Upper Class Committee\textcolor{red}{man team} will be \textcolor{ForestGreen}{the} academic year directly following their appointment, with the exception of the newly-appointed Head Upper Class Committee\textcolor{ForestGreen}{-member and Diversity Council-members}, who will begin serving in their role\textcolor{ForestGreen}{s} during Third Term.
\end{itemize}

\section{Gender Neutral Wording}
\begin{itemize}
	\item \textbf{Date Passed:} 1/19/2019
	\item \textbf{Author:} Emily Wu and Erik Herrera
	\item \textbf{Reasoning:} The current version of the constitution uses only ‘He/him/his’ pronouns. We would like to use more inclusive language. This would benefit current Ruddock House members and all the classes to come. 
	\item \textbf{Current Language:} Numerous instances. The change will apply to the entire constitution.
	\item \textbf{Proposed Language:} Change everywhere the constitution uses male-centered language to gender-neutral language. Examples include \textcolor{red}{he/him/his} to genderless pronouns (\textcolor{ForestGreen}{they/them}) or avoid pronoun use, Committee\textcolor{red}{man} to Committee\textcolor{ForestGreen}{-member}, and fresh\textcolor{red}{man} to \textcolor{ForestGreen}{frosh}. 
\end{itemize}

\section{Splitting Steward Role Between Veep and Treasurer}
\begin{itemize}
	\item \textbf{Date Passed:} 1/19/2019
	\item \textbf{Author:} Rupesh Jeyaram
	\item \textbf{Reasoning:}  As of now, Veep can assume more House responsibilities. The lack of Veep duties results from the Veep/HUCC split two years ago. Treasurer already has a lot of responsibilities, and it would be nice to distribute work more evenly. Makes sense to shift responsibilities in this manner.
	\item \textbf{Current Language:} \\
	2.2 Qualifications and Duties of Elected Officers \\
	2. The Vice President shall be a junior or senior for the latter portion of his term in office. He shall assume the duties of the President in the event of the President’s absence. He shall be responsible for maintaining the external relations of the House, for bringing guests to the House, and shall act as official host for all House guests. He shall be the Ruddock representative on the Review Committee. \\
	4. The Treasurer shall be responsible to the House for all receipts and expenditures, shall maintain an adequate book-keeping system, shall submit a report to the Executive Committee at each regularly scheduled meeting, shall propose a budget to the Executive Committee for each term, and shall submit to the House a written report of the finances at the end of each term. 
	\item \textbf{Proposed Language:} \\
	2.2 Qualifications and Duties of Elected Officers \\
	2. The Vice President shall be a junior or senior for the latter portion of his term in office. He shall assume the duties of the President in the event of the President’s absence. He shall be responsible for maintaining the external relations of the House, for bringing guests to the House, and shall act as official host for all House guests. He shall be the Ruddock representative on the Review Committee. \textcolor{ForestGreen}{They shall share duties with the Treasurer in maintaining house property.} \\
	4. The Treasurer shall be responsible to the House for all receipts and expenditures, shall maintain an adequate book-keeping system, shall submit a report to the Executive Committee at each regularly scheduled meeting, shall propose a budget to the Executive Committee for each term, and shall submit to the House a written report of the finances at the end of each term. \textcolor{ForestGreen}{They shall share duties with the Vice President in maintaining house property.}
\end{itemize}

\end{document}