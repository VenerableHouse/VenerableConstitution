\documentclass[10pt]{article} % use larger type; default would be 10pt
\usepackage[utf8]{inputenc} % set input encoding (not needed with XeLaTeX)

\usepackage{geometry} % to change the page dimensions
\geometry{letterpaper}
\usepackage{graphicx} % support the \includegraphics command and options
\setlength\voffset{-0.2in}

\title{\includegraphics[width=1.55in]{ruddock_crest.png} \\ Constitution}
\author{Ruddock House}
%\date{} % Activate to display a given date or no date (if empty),

\begin{document}

\maketitle

\section{Name and Membership}
\subsection{Name}
The name of the organization shall be Ruddock House.
\subsection{Membership}
\begin{enumerate}
\item The types and qualifications of membership are:
\begin{enumerate}
\item a Full Member, who is an undergraduate of the California Institute of Technology and is picked as such at the end of Rotation Week, or is elected as such by a 2/3 majority of those voting at a House meeting;
\item a Social Member, who is an undergraduate at the California Institute of Technology and has been elected as such, following a week of consideration, by a simple majority of those voting, when those voting constitute at least 1/3 of the House Membership;
\item an Associate Member, who was an undergraduate of the California Institute of Technology and was a Full Member or Social Member of Ruddock House, or who is any other person associated with Ruddock House and so designated by the Executive Committee;
\item the Resident Associates, who shall be considered to be non-voting Full Members.
\end{enumerate}
\item Full and social members who discontinue paying their house dues while they are registered students at the California Institute of Technology must be re-elected at a house meeting as provided for in Section 1.2, part 1a.
\item Full members can voluntarily demote themselves to social membership or cancel their membership. Social members can voluntarily cancel their membership.
\end{enumerate}
\subsection{Duties and Privileges}
\begin{enumerate}
\item The duties and privileges of Full Members apply immediately upon attaining such status. These duties and privileges are:
\begin{enumerate}
\item to attend all regular and special House Meetings;
\item to participate actively in all House functions;
\item to pay dues;
\item to have room preference;
\item to vote on all matters put before the House;
\item to abide by the rules and regulations made by the Executive Committee, the Upper Class Committee, and the Committee on Student Houses.
\end{enumerate}
\item The duties and privileges of Social Members apply immediately upon ratification by The House and are identical to Full Members, with the exception of:
\begin{enumerate}
\item the right to have room preference;
\item the right to vote on all matters put before The House;
\end{enumerate}
\item The duties and privileges of associate members:
\begin{enumerate}
\item do not include a vote in House matters;
\item shall be determined by the Executive Committee in other extents.
\end{enumerate}
\end{enumerate}
\section{Officers}
\subsection{General Qualifications}
Each officer shall be a Full Member of the House when elected and during his term of office. If, during his term of office, his membership changes, he immediately forfeits his office.
\subsection{Qualifications and Duties of Elected Officers}
\begin{enumerate}
\item The President shall reside in Ruddock House and shall be a junior or senior for the latter portion of his term in office. He shall represent the House on the Interhouse Committee, preside over House Meetings, be chairman of the Executive Committee, and be an ex-officio member of all House committees, and shall be responsible for selecting Full Members to attend freshmen picks.
\item The Vice President shall be a junior or senior for the latter portion of his term in office. He shall assume the duties of the President in the event of the President’s absence. He shall be responsible for maintaining the external relations of the House, for bringing guests to the House, and shall act as official host for all House guests. He shall be the Ruddock representative on the Review Committee.
\item The Secretary shall record minutes, including all motions made and the votes on these by all officers present, of all House, Executive Committee, and Upper Class Committee meetings; and post publicly a copy of the minutes of all open meetings. He shall have custody of all House records not specifically delegated to the custody of another officer, and in general shall perform all the duties of a corresponding and recording secretary. He shall be responsible for room assignments.
\item The Treasurer shall be responsible to the House for all receipts and expenditures, shall maintain an adequate book-keeping system, shall submit a report to the Executive Committee at each regularly scheduled meeting, shall propose a budget to the Executive Committee for each term, shall submit to the House a written report of the finances at the end of each term, and shall maintain house property.
\item The Librarian shall reside in Ruddock House and shall be a sophomore for the latter portion of his term in office. He shall be responsible for procuring magazines and newspapers, and for maintaining the House Library, including its records. He shall publish a House list at the beginning of each term. He shall be responsible for freshman orientation and organizing the freshman members of the House for the performance of those tasks assigned to them.
\item The office of Social Team shall be held by not more than five members of the House, among them a single Social Chairman. The Social Team will be responsible for organizing social functions, and for publishing a calendar of such events.
\item The office of Athletic Team shall be held by not more than three members of the House, among them a single Athletic Manager. The Athletic Team shall be responsible for all athletic equipment used by the House, for the participation of the House in Interhouse and intrahouse athletics, and in general shall organize and promote the athletic activities of the House.
\item The office of Sophomore Representative shall be held by a sophomore for the latter portion of his term in office. He shall be responsible for representing the thoughts and ideas of the sophomore members of the House on the Executive Committee.
\item The office of Off-Campus Representative shall be held by a member of the House who has resided off-campus\footnote{We denote an off-campus member of Ruddock House as a member who does not physically reside in the house.} for at least one full academic term. He shall be responsible for representing the thoughts and ideas of the Off-Campus members of the House on the Executive Committee.
\item The order of succession shall be President, Vice President, Secretary, Treasurer, Librarian, Social Chairman, Athletic Manager, Sophomore Representative, and Off-Campus Representative.
\item In the event that the President is unable to complete his term of office, the Vice President shall assume all responsibilities of the President, and a new Vice President shall be elected pursuant to nomination and election guidelines.
\end{enumerate}
\subsection{Duties and Qualifications of Appointed Officers}
\begin{enumerate}
\item One Upper Class Committeeman shall reside in each Ruddock House Alley and two shall reside off-campus, except in the case where no suitable volunteers can be found living off-campus, in which case the Off-Campus Upper Class Commiteemen may live on-campus. An Upper Class Commiteeman shall have been both enrolled at Caltech for at least five academic terms and a Full House Member for at least three academic terms prior to his term of office. A Peer Advocate shall have been both enrolled at Caltech for at least five academic terms and a Full House Member for at least three academic terms prior to his term of office. He shall be available to counsel any member of the House, particularly the occupants of the alley in which he resides or, in the case of an Off-Campus Upper Class Commiteemen, any off-campus member. He shall be responsible for maintaining order and enforcing the rules and policies of the House. He shall not concurrently be an Upper Class Committeeman and hold a position on the Executive Committee.
\item The Head Upper Class Committeeman shall have all the qualifications and duties of an Upper Class Committeeman. He shall be the head and ex officio member of the Upper Class Committee. He shall be responsible for the entire freshman selection except for the duties delegated to the President. The Head Upper Class Committeeman shall have previously served as a Ruddock Upper Class Committeeman, except in the case where no suitable volunteers can be found.
\item The Historians shall be responsible for all recording and preservation of House records and traditions, including House scrapbook, House history, and House roll, and for submission of the House article in the Institute yearbook.
\item The Head Waiter shall be responsible for maintaining proper decorum and conduct in the dining room, and shall direct the student waiters.
\item The House BFD Editors shall be responsible for publishing the House paper.
\item The House O’Domhnaill’s Suppliers shall maintain the House convenience store.
\end{enumerate}
\subsection{Nominations, Elections, and Appointments}
\begin{enumerate}
\item Nominations for ARC representative, CRC representative, two BoC representatives, Off-Campus Representative, Sophomore Representative, Athletic Manager, Athletic Team, Social Chairman, Social Team, Librarian, Treasurer, Secretary, Vice President, and President, in that order, shall be made at a meeting of the House held for that purpose between the beginning of the second term and the third Friday of second term. Nominations shall remain open until 36 hours before the start of the elections as described in section 2.4.2.
\item Election of all officers except Social Team and Athletic Team shall be carried out at a House Meeting held for that purpose within one week after the nomination meeting. The order of election shall be in the following order: President, Vice President, Secretary, Treasurer, Librarian, Social Chairman, Athletic Manager, Sophomore Representative, Off-Campus Representative, two BoC representatives, CRC representative, ARC representative. The election for each office shall be held before nominations for successive offices are closed. Election shall require a majority of votes cast from the total voting membership, not including blank ballots, and they shall be made by secret ballot. Elections shall be managed by an Election Committee in accordance with the Procedural Amendment.
\item Election for Social Team and Athletic Team shall be carried out through proxy voting beginning at the conclusion of the House Meeting described in section 2.4.2 and concluding 48 hours later. The elections shall be managed by the Election Committee.
\item All officers shall be installed at a meeting of the House held not more than one week from their election. The oath of office shall be administered by the retiring president. Each officer shall assume the duties of office immediately upon being installed.
\item The selection committee for Peer Advocates will be composed of the Head Upper Class Committeeman, the President, the most recent former President, and the highest-ranking sophomore member of the Executive Committee. If there are no sophomore members of the Executive Committee, then the Vice President will fulfill this role. If the President is serving a second term, then the highest-ranking non-freshman member of the Executive Committee not already on the selection committee shall fulfill the role of the most recent former President. 
\item Sign-up lists for appointed offices shall be posted between the first day of Third Term and the third Friday of Third Term. Sign-up lists shall be removed and appointments made by the Executive Committee two weeks thereafter. The current Head Upper Class Committeeman shall also be present and participant in the appointment process. The term of service for appointed offices shall be the academic year directly following their appointment by the Executive Committee and the Head Upper Class Committeeman. Students selected as Peer Advocates will be guaranteed positions as Upper Class Committeemen for the upcoming year, and other Upper Class Committeemen will be appointed from the candidate pool to complete the Upper Class Committeeman team.
\item Except as stated in Section 2.2, Paragraph 11, if an unexpected vacancy in an elected or appointed office occurs, the Executive Committee, or the remaining part thereof, shall select a qualified member of the House to fill the vacancy until the next election or appointment process. If an elected or appointed officer plans to be absent or otherwise unable to fulfill their duties for an extended period of time (e.g. study abroad), the officer shall notify the Executive Committee of their intended absence, and the Executive Committee shall then appoint a qualified member of the House to temporarily fill the position until the officer can assume his or her duties again.
\end{enumerate}
\subsection{Recall}
\begin{enumerate}
\item Any elected officer may be recalled by a majority vote of the voting membership of the House. 
\item The procedure shall be as follows:
\begin{enumerate}
\item Petitions, containing reasons and purpose, must be submitted in writing, endorsed by 1/3 of the total Full Membership, at a House meeting held for that purpose;
\item The petitions must be posted immediately following the presentation and remain posted until voting is closed;
\item Voting shall follow the specified procedures for Constitutional amendment elections.
\end{enumerate}
\end{enumerate}
\section{The Executive Body}
\subsection{Name}
The executive body of the House shall be known as the Executive Committee.
\subsection{Membership}
The Executive Committee shall consist of the President, who shall act as chairman, the Vice President, Secretary, Treasurer, Librarian, Social Chairman, Athletic Manager, Sophomore Representative, and Off-Campus Representative.
\subsection{Voting}
All members of the Executive Committee shall vote. Each office shall receive one vote.
\subsection{Powers}
\begin{enumerate}
\item The Executive Committee shall formulate House rules and policies, and assume all duties not otherwise delegated.
\item It shall appoint all offices described under Section 2.3 of this Constitution.
\item It shall designate associate members.
\item It shall have the power of dismissal over all appointed officers, other than Upper Class Committeemen.
\end{enumerate}
\subsection{Meetings}
\begin{enumerate}
\item Meetings may be called by the President or any two members.
\item A quorum shall consist of five of the nine offices.
\item There shall be regularly scheduled meetings, at least biweekly, the day to be determined by the Executive Committee.
\item Closed meetings, attended only by the Executive Committee members, shall be held when discussing individuals and other sensitive issues, at the discretion of the Executive Committee.
\item Upon petition of 1/3 of the Full Membership of the House, a meeting of the Executive Committee shall be held within two weeks to consider any proposal so petitioned.
\end{enumerate}
\section{The Judiciary Body}
\subsection{Name}
The judiciary body of the House shall be known as the Upper Class Committee.
\subsection{Membership}
The Upper Class Committee shall consist of the Upper Class Committeemen.
\subsection{Powers and Duties of the Committee}
The Upper Class Committee:
\begin{enumerate}
\item shall review and act on all violations of House rules and policy, by reprimand, punishment, or other appropriate action.
\item shall be responsible for the administration and tabulation of recall and amendment elections.
\item shall be autonomous in all matter of judicial policy not otherwise specified in this Constitution.
\end{enumerate}
\subsection{Meetings}
\begin{enumerate}
\item Meetings may be called by the Head Upper Class Committeeman, Resident Associate, or any two Upper Class Committeemen.
\item The Head Upper Class Committeeman shall preside over Upper Class Committee meetings and shall vote only in the case of a tie.
\item A quorum shall consist of at least 2/3 of the Upper Class Committeemen, provided that when the Committee considers the case of an individual person, the committeeman of the alley in which the person lives, and the committeeman of the alley in which the infraction occurred, must be present.
\end{enumerate}
\section{House Meetings}
\subsection{Meetings}
\begin{enumerate}
\item The House shall meet:
\begin{enumerate}
\item for the nomination, election, and recall of officers;
\item when a meeting is called by the Executive Committee;
\item at other times, on petition by 1/5 of the Full Membership.
\end{enumerate}
\item Notice of the meeting shall be published at least three days prior to each House meeting in a manner expected to reach the entire membership of the House.
\begin{enumerate}
\item The notice shall contain the date, time, and place of the meeting and an agenda of all issues to be discussed and voted upon at that meeting.
\item No substantive issues not on a meeting’s posted agenda may be voted upon at that meeting. Discussion of any issue and voting on non-substantive issues are not constrained.
\end{enumerate}
\end{enumerate}
\subsection{Quorum}
\begin{enumerate}
\item A quorum shall consist of the larger of:
\begin{enumerate}
\item 1/3 of the Full-Membership of the House, or
\item 80\% of the average attendance of House meetings held during the three hundred sixty-five (365) days prior to the announced date of the meeting regardless of whether those meetings established quorum.
\end{enumerate}
\item If quorum is not established, no substantive matters can be decided.
\end{enumerate}
\subsection{Procedure}
In case of a dispute on parliamentary procedure in meetings of the House or any of its committees, Robert’s Rules of Order (revised edition) shall provide the rule whenever the Constitution does not.
\section{Finances}
\subsection{Dues}
\begin{enumerate}
\item The dues of the House shall be due and payable by each regular member at registration for each term. 
\item Married students shall be exempt from dues.
\item The dues for Social Members shall be set by a majority vote of the Executive Committee.
\item The dues for Associate Members shall be set by a majority vote of the Executive Committee.
\end{enumerate}
\subsection{Levies}
The dues may be changed, or addition assessments levied, only by a 1/2 vote of the Full Membership.
\subsection{Payment of Moneys}
\begin{enumerate}
\item Checks and other instruments for the payment of moneys shall be drawn in the name of the House and shall be signed either by the President or by the Treasurer.
\item The Treasurer may authorize House related expenditures up to \$250.
\item House related expenditures greater than \$250 must be approved by a simple majority vote of the Executive Committee.
\item Any expenditure may be placed to a House vote at the discretion of the Executive Committee. A simple majority of the House voting membership may approve any expenditure.
\end{enumerate}
\section{Room Hassles}
\subsection{Procedure}
\begin{enumerate}
\item The Secretary shall be responsible for running the Room Hassle in the spirit of fairness to maximize the occupancy of the house.
\item A Room Hassle shall be held:
\begin{enumerate}
\item once per year following the announcement of appointed offices to allocate rooms for the next academic year
\item whenever beds are available and there are Full Members seeking residence
\item whenever a partially filled room is sought by a member higher in the order given by section 7.1.4.
\end{enumerate}
\item A Room Hassle shall continue until either all the available beds have been filled or there are no longer any Full Members seeking residence.
\item Rooms, with the exception of a designated sophomore off-campus alley, shall be picked in the following order:
\begin{enumerate}
\item by the House President, then
\item by all other Full Members in order of
\begin{enumerate}
\item class (seniors then juniors...), then
\item intended number of occupants as a percentage of the room capacity, descending, then
\item office as specified in section 7.2.8, then
\item up to two random sophomores as determined by a random order prepared prior to the section 7.1.2a Room Hassle in a manner prescribed by the Secretary, then
\item a random order prepared prior to the section 7.1.2a Room Hassle in a manner prescribed by
the Secretary.
\end{enumerate}
\end{enumerate}
\item Prior to the section 7.1.2a Room Hassle the Secretary shall designate an off-campus alley\footnote{The framers of this Constitution recommend 260 Chester as the designated off-campus alley.} for the members of the sophomore class. The residency shall be filled in the following manner:
\begin{enumerate}
\item intended number of occupants, then
\item intended number of sophomore occupants, then
\item sophomores’ office as specified in section 7.2.8, then
\item sophomores’ random order prepared prior to the section 7.1.2a Room Hassle in a manner proscribed by the Secretary.
\end{enumerate}
\item In the event that no members of the sophomore class wish to reside within the off-campus alley, the standard pick order shall be used. 
\item The Upper Class Committee shall arbitrate all conflicts and disputes.
\item The Secretary may consolidate unoccupied beds if doing so will increase the number of occupants of the house.
\end{enumerate}
\subsection{Guarantees}
\begin{enumerate}
\item All Full Members shall have the opportunity to participate in any Room Hassle in person or by proxy.
\item All Full Members shall have the opportunity to designate their roommate(s) before another individual picks or is placed into the room. Thereafter, any unoccupied bed may be picked.
\item Contracted residents may exchange rooms and beds with each other at their own discretion. In the event of a dispute between roommates, the member with the higher pick shall have preference.
\item Room Hassles under section 7.1.2b or section 7.1.2c may not forcibly displace any participant from the house, and all non-participants shall retain their current rooms except as in 7.1.8. Members holding rooms sought under section 7.1.2c must participate in the hassle.
\item At any time during a Room Hassle, any participant may restart the process from his position in the order.
\item The President may have himself counted as a double occupant.
\item Entire rooms shall be guaranteed to the following officers:
\begin{enumerate}
\item In order of preference, President, Vice President, Secretary, Treasurer, Librarian, Social Team including the Social Chairman (up to three members), Athletic Team including the Athletic Manager (up to two members), Head UCC, On Campus UCCs (one member per alley), Peer Advocates, House-elected BoC Representatives (up to two members), O’Domhnaill’s Suppliers (up to two members), House Head Waiter (one member), House BFD editors (one member), House Historians (one member), IHC Chairman.
\item for the entire academic year following the election or appointment provided the officer completes his term. If an officer quits his office mid-term, the room guarantee shall transfer to the succeeding officeholder for the remainder of the guarantee term.
\end{enumerate}
\item The officer room pick order shall be President, Vice President, Secretary, Treasurer, Librarian, Social Chairman, Athletic Manager, Head UCC, On Campus UCCs, Social Team, Athletic Team, House-elected BoC Representatives (up to two members), O’Domhnaill’s Suppliers (up to two members), House Head Waiter (one member), House BFD editors (one member), House Historians (one member), IHC Chairman.
\item All Full Members who have picked into a room through a room hassle under 7.1.2a or 7.1.2b shall be guaranteed the same room upon their return from Study Abroad.
\end{enumerate}
\section{Amendments}
\subsection{Order}
Only Full Members may vote on Constitutional amendments.
\subsection{Necessary Vote}
\begin{enumerate}
\item In the case of non-conflicting amendments, an amendment to this Constitution passes if at least 2/3 of the votes cast are in favor of the amendment.
\item In the case of conflicting amendments being voted on concurrently, votes shall be cast as if they are for a position in the house. All amendments failing to reach a 2/3 supermajority approval over NO (indicating no change; NO shall be ranked as if it is a candidate) shall be thrown out, and the ranked pairs (Tideman) method shall be applied to any remaining amendments to determine which amendment passes. Such conflicting amendments must be proposed simultaneously; concurrent voting on conflicting amendments can only be done if voting begins and ends at the same time for all such amendments. In the case of a tie between multiple amendments, all amendments shall be reproposed to the house, discussed at another meeting, and revoted on. NO wins in the case of a tie.
\end{enumerate}
\subsection{Procedure}
\begin{enumerate}
\item Proposed amendments, containing reasons, purpose, and un-amended and amended language must be submitted electronically to the Full Membership of the House, endorsed by 1/5 of the Full Membership or two members of the Executive Committee not including the proposer, at least three days prior to a House Meeting.
\item The proposed amendments must be posted immediately following their presentation and remain posted until voting is closed.
\item Voting
\begin{enumerate}
\item Voting shall officially commence upon the legal presentation of the amendment and shall close at 11:59 PM on the seventh day following;
\item Voting shall be by secret ballot;
\item Votes of House Members shall not be examined prior to the closing of the election;
\item A roll of those members who have voted shall be available throughout the election.
\end{enumerate}
\end{enumerate}

\pagebreak
\appendix
\section{Procedural Amendment}
An Election Committee shall be appointed at the nomination meeting. The Election Committee shall be composed of nine Full House Members: 3 sophomores, 3 juniors and 3 seniors, who shall be randomly selected from a pool of volunteers who are not running for an elected office. In the event that too few members of a specific class volunteer, the remaining members shall be drawn from the other classes.

Paper ballots shall be cast anonymously at a House Meeting held for that purpose. The Election Committee shall be responsible for collecting and tabulating ballots. The winner of each election, other than for the BoC representatives, shall be determined as follows: all candidates that do not achieve a majority over ``NO'' are eliminated, and the ranked pairs (Tideman) method is applied to the remaining candidates. The Election Committee shall be responsible for breaking any ties that occur during this process.

For the election of the BoC representatives, each voter may cast two first choice votes and rank their replacements in the case that one of their first choices is eliminated. Votes shall be counted via elimination of the candidate with the lowest number of first choice votes until two candidates receive over 1/3 of the votes each.

Proxy votes shall be submitted to a member of the Election Committee in written form prior to the election. The Election Committee shall be responsible for entering proxy votes during the election.

The Election Committee shall hear the grievance of any Full House Member from the time of the election to 36 hours post. They shall have closed meetings to determine the legitimacy of the grievance and shall announce any solution to the House. If a solution cannot be reached by the Election Committee, then the election shall be rerun, in part or in full, in accordance with the nature of the grievance. If the grievance is regarding the persons on the Election Committee, then the Upperclassmen Committee shall hear the grievance. They shall have the powers to disband the Election Committee, force a redrawing of a new Election Committee, and begin the election process anew.
\end{document}
